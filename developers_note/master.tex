\documentclass[uplatex, a4paper]{jsreport}
\usepackage[dvipdfmx]{graphicx, xcolor}

\usepackage[dvipdfmx]{hyperref} % ハイパーリンク
\usepackage{pxjahyper}          % 日本語の文字化けを防ぐ
\hypersetup{
  colorlinks = true,
  citecolor  = blue,
  linkcolor  = blue,
  urlcolor   = blue,
}

\usepackage{tabularx}   % 表
\usepackage{layout}     % レイアウト

\usepackage{amsmath, amssymb} % 数式と数学記号
\usepackage{bm}               % ボールド体

\usepackage{url}      % URLを書くため
\usepackage{siunitx}  % SI単位を書くため

\usepackage{tikz}     % TikZ
\usetikzlibrary{intersections, calc, arrows}

\usepackage{listings} % コードブロック
\lstset{
  basicstyle = \small\ttfamily,
  keywordstyle = \color[RGB]{33, 150, 243},
  commentstyle = \color[RGB]{76,175,80},
  backgroundcolor = \color[gray]{.95},
  breaklines = true,
  breakautoindent = true,
  captionpos = t,
  keepspaces = true,
  columns = flexible,
  showspaces = false,
  showstringspaces = false,
  showtabs = false,
  aboveskip = 6pt,
  belowskip = 6pt,
  xleftmargin = 10pt,
  xrightmargin = 10pt,
  frame = tb,
  framesep = 8pt,
  framerule = 0.1pt,
  frameround = tttt,
  framexleftmargin = 10pt,
  framexrightmargin = 10pt,
  rulecolor = \color[gray]{.95},
  tabsize = 2
}

% URDF言語のシンタックスハイライトのキーワードを追加
\lstdefinelanguage{XML}
{
  morestring=[s]{"}{"},
  morestring=[s]{>}{<},
  morecomment=[s]{<?}{?>},
  morecomment=[s]{!--}{-->},
  stringstyle=\color{black},
  identifierstyle=\color[RGB]{33, 150, 243},
  keywordstyle=\color[RGB]{124, 77, 255},
  morekeywords={
    name,
    type,
    xyz,
    rpy,
    filename,
    rgba,
    params,
    default,
    command,
    pkg,
    respawn,
    output,
    args,
    file
  }
}

% YAML言語のシンタックスハイライト
\lstdefinelanguage{yaml}
{
  basicstyle = \small\ttfamily\color[RGB]{76,175,80},
  morecomment = [l]{\#},
  commentstyle = \color[RGB]{117, 117, 117},
  moredelim = [l][{\color[RGB]{33, 150, 243}}]{:}
}

\lstdefinelanguage{cmake}{
  morekeywords={
    cmake_minimum_required,
    project,
    find_package,
    find_library,
    catkin_package,
    include_directories,
    add_executable,
    target_link_libraries,
  }
}

\usepackage{subfiles} % ファイル分割

% ---レイアウトパラメータ---
\usepackage[left=25mm,right=25mm,top=20mm,bottom=20mm, includefoot]{geometry}

\pagestyle{plain}   % ページ番号表示

\renewcommand{\tablename}{Table}    % 表番号フォーマットをTableへ
\renewcommand{\figurename}{Fig.}    % 図番号フォーマットをFig.へ

\renewcommand{\lstlistingname}{Code}  % コードブロックのキャプションを"Code"に変更

\renewcommand{\baselinestretch}{0.8}    % 行間の調整
\renewcommand{\arraystretch}{1.2}       % 表の行の高さの調整

\title{\Huge Developers Note for ADAMR2 Project} 
\author{一関高専 専攻科 生産工学専攻 2年 藤野航汰}
\date{\today}

% ---ドキュメント開始---
\begin{document}

\maketitle  % タイトル生成

\tableofcontents  % 目次生成
\newpage

\chapter{ハードウェア組立に関するTips}
\subfile{sections/hardware/solderless_terminal.tex}

\chapter{URDFを記述する}
\subfile{sections/urdf_modeling/what_is_urdf.tex}
\subfile{sections/urdf_modeling/why_urdf.tex}
\subfile{sections/urdf_modeling/target_overview.tex}
\subfile{sections/urdf_modeling/create_package.tex}
\subfile{sections/urdf_modeling/add_base_link.tex}
\subfile{sections/urdf_modeling/add_caster_link.tex}
\subfile{sections/urdf_modeling/add_lidar_link.tex}
\subfile{sections/urdf_modeling/add_wheel_link.tex}

\chapter{デバイス設定}
\subfile{sections/devices/logicool_joystick.tex}
\subfile{sections/devices/rplidar_a2.tex}

\chapter{コントローラの実装}
\label{chap:controller_implementation}
\subfile{sections/controller/what_is_ros_control.tex}
\subfile{sections/controller/required_things_for_ros_control.tex}
\subfile{sections/controller/create_adamr2_control_package.tex}
\subfile{sections/controller/controller_launch_file.tex}
\subfile{sections/controller/config_for_diff_drive_controller.tex}
\subfile{sections/controller/ypspur_param_file.tex}
\subfile{sections/controller/write_ypspur_launcher.tex}
\subfile{sections/controller/create_adamr2_driver_package.tex}
\subfile{sections/controller/launch_controller.tex}

\chapter{SLAMによる地図作成}
\subfile{sections/slam/slam_gmapping.tex}
\subfile{sections/slam/required_things_for_slam.tex}
\subfile{sections/slam/create_package.tex}
\subfile{sections/slam/procedure_of_slam.tex}

\bibliography{bibliography}
\bibliographystyle{junsrt}
\end{document}
% --- ドキュメント終了 ---