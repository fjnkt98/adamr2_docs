\documentclass[{../../master}]{subfiles}
\graphicspath{{../../}}  % 個別コンパイル時の画像パスを解決する

\begin{document}

\section{Navigationのためのパッケージ作成}

Navigationを実行するためのパッケージを作成します.
パッケージ名は\textsf{adamr2\_navigation}とします.
コード\ref{code:create_adamr2_navigation_package}を実行して,パッケージを作成します.
依存パッケージの指定はここではしなくても構いません.

\begin{lstlisting}[language=sh, label=code:create_adamr2_navigation_package, caption=Create \textsf{adamr2\_navigation} Package]
catkin create pkg adamr2_navigation
\end{lstlisting}

このパッケージにはNavigation Stackのパッケージのノードを実行するためのlaunchファイルと,各種ノード・プラグインに与えるパラメータを記述したパラメータファイルを格納します.
ディレクトリ構成は\ref{code:directory_structure_of_adamr2_navigation]
adamr2_description/}のようになります.

\begin{lstlisting}[language=sh, caption=Directory Structure of \textsf{adamr2\_navigation}, label=code:directory_structure_of_adamr2_navigation]
adamr2_navigation/
  ├ launch/
  ├ config/
  ├ package.xml
  └ CMakeLists.txt
\end{lstlisting}

\textsf{launch/}ディレクトリにlaunchファイルを,\textsf{config/}ディレクトリにパラメータファイルを置いていくのですが,Navigation Stackのパラメータは多岐に渡るので,ファイル数やディレクトリ構造が複雑になります.
次の節でパラメータファイルの書き方を,またその次の節でlaunchファイルの書き方を説明します.

\end{document}