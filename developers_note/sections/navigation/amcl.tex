\documentclass[{../../master}]{subfiles}
\graphicspath{{../../}}  % 個別コンパイル時の画像パスを解決する

\begin{document}

この章ではNavigation Stackの各パッケージのパラメータのチューニング方法について解説します.

移動ロボットの自律走行を適切に行うためには,ロボットの仕様や動作環境等に合わせた最適なパラメータを設定する必要があります.
しかし,パラメータチューニングを行うためには,各々のノードがどのようなアルゴリズムで処理を行っているかをある程度把握している必要があります.
本章ではNavigation Stackの各ノードのパラメータチューニングを行うための情報について,アルゴリズムやデータ構造について言及しながら解説していきます.

尚,本章の解説は筆者が理解できた範囲での説明となります.
ここで解説する内容が絶対的に正しいという保証はありません.
間違いや説明抜け等が存在するため,あくまで参考程度にするようにしてください.

\section{\textsf{amcl}について}

\textsf{amcl}は与えられた地図,ホイールオドメトリ情報,レーザースキャンデータからMonte Carlo Localizationによってロボットの自己位置推定を行うパッケージです.
\textsf{amcl}ではKLDサンプリングによるリサンプリング\cite{amcl}とAugmented MCL\cite{thrun2005probabilistic}が実装されていて,前者によってパーティクル数を可変にし,後者によってセンサリセットを実現しています.

\subsection{\textsf{amcl}の入力と出力}

\textsf{amcl}はレーザースキャン(\textsf{sensor\_msgs/LaserScan}型メッセージ)と,オドメトリ座標系からロボットのベースフレームまでのTF(デフォルトでは``odom''から``base\_link''の間のTF)を入力として受け取り,推定結果として地図座標系からオドメトリ座標系までのTFを出力します.

オドメトリとして必要なのはあくまでTFであり,トピックメッセージとしてのオドメトリ(\textsf{geometry\_msgs/Odometry}型メッセージ)は必要ありません.

\subsection{\textsf{amcl}のパラメータ}

ROS Wikiより,\textsf{amcl}ノードが使用するパラメータを列挙します.

\subsubsection{フィルタ全般に関するパラメータ}

\begin{itemize}
  \item \textsf{min\_particle} \\
    int型,default: 100 \\
    パーティクルフィルタで使用するパーティクルの最低数です.
  \item \textsf{max\_particle} \\
    int型,default: 5000 \\
    パーティクルフィルタで使用するパーティクルの最大数です.
  \item \textsf{kld\_err} \\
    double型,default: 0.01 \\
    KLDサンプリングにおける,真の信念分布と推定信念分布との誤差の最大許容量です.このパラメータを小さくすると自己位置推定の精度は向上しますが,パーティクル数が増大して計算量が大きくなります.
  \item \textsf{kld\_z} \\
    double型,default: 0.99 \\
    KLDサンプリングにおける,「信念分布の誤差が\textsf{kld\_err}未満になる確率$p$」に対する確率$1-p$の標準正規分布の分位数です.このパラメータを大きく設定すると,誤差が許容量未満になる確率が大きくなるため,必要なパーティクル数が増大します.
  \item \textsf{update\_min\_d} \\
    double型,default: 0.2 \\
    パーティクルフィルタの更新を行うために必要な最小並進移動距離です.この距離を移動しなければパーティクルフィルタは更新されません.
  \item \textsf{update\_min\_a} \\
    double型,default: $\pi /6$ \\
    パーティクルフィルタの更新を行うために必要な最小旋回移動距離です.この距離を移動しなければパーティクルフィルタは更新されません.
  \item \textsf{resample\_interval} \\
    パーティクルのリサンプリングを行うために必要なフィルタの更新回数です.ここで指定した回数だけフィルタが更新されなければリサンプリングは実行されません.
  \item \textsf{transform\_tolerance} \\
    double型,default: 0.1 \\
    \textsf{amcl}が推定した自己位置(\textsf{map} $\rightarrow$ \textsf{odom}のTF)のタイムスタンプを,ここで指定した時間だけ実際の時刻より未来にします.
  \item \textsf{recovery\_alpha\_slow} \\
    double型,default: 0.0 \\
    Augmented MCLにおける,周辺尤度$\alpha$の平均重みフィルタの減衰率です.
  \item \textsf{recovery\_alpha\_fast} \\
    double型,default: 0.0 \\
    Augmented MCLにおける,周辺尤度$\alpha$の平均重みフィルタの減衰率です.
  \item \textsf{initial\_pose\_x} \\
    double型,default: 0.0 \\
    初期姿勢の$x$座標の平均値です.正規分布とともにフィルタの初期化に使用されます.
  \item \textsf{initial\_pose\_y} \\
    double型,default: 0.0 \\
    初期姿勢の$y$座標の平均値です.正規分布とともにフィルタの初期化に使用されます.
  \item \textsf{initial\_pose\_a} \\
    double型,default: 0.0 \\
    初期姿勢の向きの平均値です.正規分布とともにフィルタの初期化に使用されます.
  \item \textsf{initial\_cov\_xx} \\
    double型,default: 0.0 \\
    初期姿勢の$x$座標の共分散です.正規分布とともにフィルタの初期化に使用されます.
  \item \textsf{initial\_cov\_yy} \\
    double型,default: 0.0 \\
    初期姿勢の$y$座標の共分散です.正規分布とともにフィルタの初期化に使用されます.
  \item \textsf{initial\_cov\_aa} \\
    double型,default: 0.0 \\
    初期姿勢の向きの共分散です.正規分布とともにフィルタの初期化に使用されます.
  \item \textsf{gui\_publish\_rate} \\
    double型,default: -1.0 \\
    レーザースキャンと軌跡を可視化する際の最大配信周期です.負の値を設定するとデータの配信が無効化されます.
  \item \textsf{save\_pose\_rate} \\
    double型,default: 0.5 \\
    最後に推定された姿勢を\textsf{initial\_pose\_*}と\textsf{initial\_cov\_*}としてパラメータサーバーに保存する頻度の最大値です.保存された姿勢は,後にもう一度\textsf{amcl}を実行する際にフィルタを初期化するために使用されます.
  \item \textsf{use\_map\_topic} \\
    bool型,default: \textsf{false} \\
    このパラメータを\textsf{true}に設定すると,\textsf{amcl}は地図を受信するためにサービスコールを行うのではなく,トピックを購読するようになります.
  \item \textsf{first\_map\_only} \\
    bool型,default: \textsf{false} \\
    このパラメータを\textsf{true}に設定すると,\textsf{amcl}は新しい地図を受信するたびに地図を更新するのではなく,最初に購読した地図のみを使用するようになります.
  \item \textsf{selective\_resampling} \\
    bool型,default: \textsf{false} \\
    このパラメータを\textsf{true}に設定すると,GMappingで採用されている$N_{\text{eff}}$の値を基準としたリサンプリング処理が行われるようになります.
\end{itemize}

\subsubsection{レーザーモデルに関するパラメータ}

\textsf{amcl}ではレーザースキャンの観測モデルとして,``beam model''と``likelihood-field model''の2つをサポートしています.
各パラメータは各モデルの確率分布の重み付け係数と形状パラメータです.

``beam model''では\textsf{z\_hit},\textsf{z\_short},\textsf{z\_max},\textsf{z\_rand}の4つが使用されます.
``likelihood-field model''では\textsf{z\_hit}と\textsf{z\_rand}の2つのみが使われます.
どちらのモデルを使う場合でも,各重み係数は和が1.0になるようにしなければなりません.

\begin{itemize}
  \item \textsf{laser\_min\_range} \\
    double型,default: -1.0 \\
    レーザーの最小距離を指定します.負の値が設定された場合,\textsf{sensor\_msgs/LaserScan}メッセージに含まれている値を使用します.
  \item \textsf{laser\_max\_range} \\
    double型,default: -1.0 \\
    レーザーの最大距離を指定します.負の値が設定された場合,\textsf{sensor\_msgs/LaserScan}メッセージに含まれている値を使用します.
  \item \textsf{laser\_max\_beams} \\
    int型,default: 30 \\
    1スキャン当たりに使用するレーザービームの数を指定します.デフォルトの設定だと,1スキャンのうち等間隔に30個のビームだけ抜き出してフィルタ処理に使用することになります.
  \item \textsf{laser\_z\_hit} \\
    double型,default: 0.95 \\
    観測モデルのうち,正規分布成分の重み係数です.
  \item \textsf{laser\_z\_short} \\
    double型,default: 0.1 \\
    観測モデルのうち,指数分布成分の重み係数です.
  \item \textsf{laser\_z\_max} \\
    double型,default: 0.05
    観測モデルのうち,無限値観測に起因する一様分布成分の重み係数です.
  \item \textsf{laser\_z\_rand} \\
    double型,default: 0.05 \\
    観測モデルのうち,ランダム観測に起因する一様分布成分の重み係数です.
  \item \textsf{laser\_sigma\_hit} \\
    double型,default: 0.2 \\
    正規分布に従うノイズの標準偏差です.単位はメートルです.
  \item \textsf{laser\_lambda\_short} \\
    double型,default: 0.1 \\
    指数分布に従うノイズのパラメータです.
  \item \textsf{laser\_likelihood\_max\_dist} \\
    double型,default: 2.0 \\
    尤度場モデルにおいて,地図上の障害物からどれだけの距離を膨張させるかを指定します.
  \item \textsf{laser\_model\_type} \\
    string型,default: ``likelihood\_field''  \\
    観測モデルの種類を指定します.\textsf{beam},\textsf{likelihood\_field},\textsf{likelihood\_field\_prob}の3つから指定できます.
    \textsf{likelihood\_field\_prob}は\textsf{likelihood\_field}とほぼ同じですが,有効化されている場合はビームスキップ処理を行います.
  \item \textsf{do\_beamskip} \\
    bool型,default: false \\
    ビームスキップ処理を行うかどうかを指定します.
  \item \textsf{beam\_skip\_distance} \\
    double型,default: 0.5 \\
    ビームが地図とマッチしているかどうかを判断するための閾値です.
    ビーム終点と一番近い障害物セルとの距離がこの閾値よりも大きいとき,そのビームは地図とマッチしないと判断されます.
  \item \textsf{beam\_skip\_threshold} \\
    double型,default: 0.3 \\
    ビームがスキップされる基準の閾値です.ビームスキップ処理では,あるビームに対して,ビームが地図とマッチしないパーティクルが十分存在するときビームがスキップされます.
    全てのパーティクルに対し,ビームがマッチしないパーティクルが一定の割合以上のとき,ビームスキップが行われます.
    その割合を決定するのがこのパラメータです.
  \item \textsf{beam\_skip\_error\_threshold} \\
    double型,default: 0.9 \\
    あまりにも多くのビームがスキップされた場合,\textsf{amcl}は逆に全てのビームを処理するようになります.
    スキップ対象のビームの数が,フィルタ更新に使用する全てのビームの数(\textsf{laser\_max\_beams}で指定した数)に対して一定割合以上になったときにビームスキップ処理を破棄します.
    その割合を決定するのがこのパラメータです.
\end{itemize}

\subsubsection{オドメトリモデルに関するパラメータ}

\begin{itemize}
  \item \textsf{odom\_model\_type} \\
    string型,default: ``diff'' \\
    オドメトリモデルで使用するロボットの移動機構の種別を指定します.
    指定できるタイプは``diff'',``omni'',``diff-corrected'',``omni-corrected''の4つです.
    ``corrected''が付いたモデルはバグ修正がされたもので,古いタイプは互換性のために残されています.
    従って,新規で作る場合は修正されたモデルを使用するべきでしょう.
    ただし,ROS Wikiに記述されているように,オドメトリモデルの各係数のデフォルト値は古いモデルにのみフィットする値であり,修正されたモデルではもっと小さい値を使うべきとされています.
    パラメータチューニングを行う際は,ロボットを実際に走らせ,オドメトリの誤差がどれくらい生じるかをもとにして行わなければなりません.
  \item \textsf{odom\_alpha1} \\
    double型,default: 0.2 \\
    ロボットの回転動作によって,オドメトリの回転成分にどれくらいのノイズが発生するかを指定します.
  \item \textsf{odom\_alpha2} \\
    double型,default: 0.2 \\
    ロボットの直進動作によって,オドメトリの回転成分にどれくらいのノイズが発生するかを指定します.
  \item \textsf{odom\_alpha3} \\
    double型,default: 0.2 \\
    ロボットの直進動作によって,オドメトリの直進成分にどれくらいのノイズが発生するかを指定します.
  \item \textsf{odom\_alpha4} \\
    double型,default: 0.2 \\
    ロボットの回転動作によって,オドメトリの直進成分にどれくらいのノイズが発生するかを指定します.
  \item \textsf{odom\_alpha5} \\
    double型,default: 0.2 \\
    ロボットの並進に関係するノイズパラメータです.``omni''モデルでのみ使用されます.
  \item \textsf{odom\_frame\_id} \\
    string型,default: ``odom'' \\
    オドメトリ座標系のフレーム名を指定します.
  \item \textsf{base\_frame\_id} \\
    string型,default: ``base\_link'' \\
    ロボットのベースリンクの名前を指定します.
  \item \textsf{global\_frame\_id} \\
    string型,default: ``map'' \\
    地図座標系のフレーム名を指定します.
\end{itemize}

\subsection{KLDサンプリング}

\textsf{amcl}はKLDサンプリング\cite{amcl}というアルゴリズムを用いることで,パーティクルの数を可変にしています.
KLDサンプリングは,ロボットの姿勢の真の信念分布とパーティクルによって近似した分布との誤差を数値化し,誤差が閾値$\varepsilon$を超えてしまう確率が$\delta$以内に収まるように,パーティクルの数を決定します.

真の信念分布とパーティクルによる分布との誤差を数値化するために,カルバック・ライブラー情報量(Kullback-Leibler divergence)というものが使用されます.
KLDサンプリングの具体的なアルゴリズムの導出については,筆者の理解が追いついていないのでここでは説明できません.
元となった論文を読むか,詳解確率ロボティクス\cite{上田2019}第7章に詳細が記述されているので,そちらを参考にして下さい.

まず,カルバック・ライブラー情報量(KL情報量)の概要について(筆者が理解できた範囲で)説明します.
KL情報量は,状態空間$\chi$中の2つの確率密度関数$p$,$q$に対して,

\begin{equation}
  \begin{split}
    D_{\text{KL}}(p \mid\mid q) &= \int_{\bm{x} \in \chi} p(\bm{x}) \log{\frac{p(\bm{x})}{q(\bm{x})}}d\bm{x} \\
    &= \langle \log{\frac{p(\bm{x})}{q(\bm{x})}} \rangle_{p(\bm{x})} \\
    &= \langle \log{p(\bm{x})} \rangle_{p(\bm{x})} - \langle \log{q(\bm{x})} \rangle_{p(\bm{x})}
  \end{split}
\end{equation}

\noindent
と定義されます.
ただし,$\langle z \rangle_{p(z)}$は確率分布$p(z)$に従う確率変数$z$の期待値で,$E_{p(z)}[z]$とも表記されます.
離散的な状態空間$M$中の確率分布$P$,$Q$の場合でも同様に,

\begin{equation}
  D_{\text{KL}}(P \mid\mid Q) = \sum_{s \in M} P(s)\log{\frac{P(s)}{Q(s)}} = \langle \log{P(s)} \rangle_{P(s)} - \langle \log{Q(s)} \rangle_{P(s)}
\end{equation}

\noindent
と定義されます.
ただし,$s$は離散的な状態空間の中の一区画で,ビンと呼ばれるものです.

2つの確率分布$p$,$q$もしくは$P$,$Q$が一致するとき,対数の中の分数が1になるため,KL情報量は0になります.
それ以外の場合は,KL情報量は常に正の値を取ります.

\textsf{amcl}では状態空間を離散化して考えます.
2次元平面を移動するロボットが取る姿勢は$\bm{x} = (x, y, \theta)^T$の3次元なので,状態空間は3次元になります.
それぞれの軸を等間隔に区切ると,状態空間は網目になります.
そのうちの1区画を$s$とし,それぞれの区画を区別するため,$s_0, s_1, \ldots, s_{k-1}$のようにインデックスを付与します.
ビンの数は$k$個になります.

そして,各ビン$s_j$内に真の姿勢$\bm{x}^*$が存在する確率を考えます.

\subsection{Augmented MCL}

\subsection{レーザースキャンのビームモデル}

\subsection{レーザースキャンの尤度場モデル}

\subsection{ビームスキップ処理}

\subsection{パラメータ調整の勘所}

\end{document}