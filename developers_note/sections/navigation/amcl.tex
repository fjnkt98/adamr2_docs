\documentclass[{../../master}]{subfiles}
\graphicspath{{../../}}  % 個別コンパイル時の画像パスを解決する

\begin{document}

この章ではNavigation Stackの各パッケージのパラメータのチューニング方法について解説します.

移動ロボットの自律走行を適切に行うためには,ロボットの仕様や動作環境等に合わせた最適なパラメータを設定する必要があります.
しかし,パラメータチューニングを行うためには,各々のノードがどのようなアルゴリズムで処理を行っているかをある程度把握している必要があります.
本章ではNavigation Stackの各ノードのパラメータチューニングを行うための情報について,アルゴリズムやデータ構造について言及しながら解説していきます.

尚,本章の解説は筆者が理解できた範囲での説明となります.
ここで解説する内容が絶対的に正しいという保証はありません.
間違いや説明抜け等が存在するため,あくまで参考程度にするようにしてください.

\section{\textsf{amcl}について}

\textsf{amcl}は与えられた地図,ホイールオドメトリ情報,レーザースキャンデータからMonte Carlo Localizationによってロボットの自己位置推定を行うパッケージです.
\textsf{amcl}ではKLDサンプリングによるリサンプリング\cite{amcl}とAugmented MCL\cite{thrun2005probabilistic}が実装されていて,前者によってパーティクル数を可変にし,後者によってセンサリセットを実現しています.

\subsection{\textsf{amcl}の入力と出力}

\textsf{amcl}はレーザースキャン(\textsf{sensor\_msgs/LaserScan}型メッセージ)と,オドメトリ座標系からロボットのベースフレームまでのTF(デフォルトでは``odom''から``base\_link''の間のTF)を入力として受け取り,推定結果として地図座標系からオドメトリ座標系までのTFを出力します.

オドメトリとして必要なのはあくまでTFであり,トピックメッセージとしてのオドメトリ(\textsf{geometry\_msgs/Odometry}型メッセージ)は必要ありません.

\subsection{\textsf{amcl}のパラメータ}

ROS Wikiより,\textsf{amcl}ノードが使用するパラメータを列挙します.

\subsubsection{フィルタ全般に関するパラメータ}

\begin{itemize}
  \item \textsf{min\_particle} \\
    int型,default: 100 \\
    パーティクルフィルタで使用するパーティクルの最低数です.
  \item \textsf{max\_particle} \\
    int型,default: 5000 \\
    パーティクルフィルタで使用するパーティクルの最大数です.
  \item \textsf{kld\_err} \\
    double型,default: 0.01 \\
    KLDサンプリングにおける,真の信念分布と推定信念分布との誤差の最大許容量です.このパラメータを小さくすると自己位置推定の精度は向上しますが,パーティクル数が増大して計算量が大きくなります.
  \item \textsf{kld\_z} \\
    double型,default: 0.99 \\
    KLDサンプリングにおける,「信念分布の誤差が\textsf{kld\_err}未満になる確率$p$」に対する確率$1-p$の標準正規分布の分位数です.このパラメータを大きく設定すると,誤差が許容量未満になる確率が大きくなるため,必要なパーティクル数が増大します.
  \item \textsf{update\_min\_d} \\
    double型,default: 0.2 \\
    パーティクルフィルタの更新を行うために必要な最小並進移動距離です.この距離を移動しなければパーティクルフィルタは更新されません.
  \item \textsf{update\_min\_a} \\
    double型,default: $\pi /6$ \\
    パーティクルフィルタの更新を行うために必要な最小旋回移動距離です.この距離を移動しなければパーティクルフィルタは更新されません.
  \item \textsf{resample\_interval} \\
    パーティクルのリサンプリングを行うために必要なフィルタの更新回数です.ここで指定した回数だけフィルタが更新されなければリサンプリングは実行されません.
  \item \textsf{transform\_tolerance} \\
    double型,default: 0.1 \\
    \textsf{amcl}が推定した自己位置(\textsf{map} $\rightarrow$ \textsf{odom}のTF)のタイムスタンプを,ここで指定した時間だけ実際の時刻より未来にします.
  \item \textsf{recovery\_alpha\_slow} \\
    double型,default: 0.0 \\
    Augmented MCLにおける,周辺尤度$\alpha$の平均重みフィルタの減衰率です.
  \item \textsf{recovery\_alpha\_fast} \\
    double型,default: 0.0 \\
    Augmented MCLにおける,周辺尤度$\alpha$の平均重みフィルタの減衰率です.
  \item \textsf{initial\_pose\_x} \\
    double型,default: 0.0 \\
    初期姿勢の$x$座標の平均値です.正規分布とともにフィルタの初期化に使用されます.
  \item \textsf{initial\_pose\_y} \\
    double型,default: 0.0 \\
    初期姿勢の$y$座標の平均値です.正規分布とともにフィルタの初期化に使用されます.
  \item \textsf{initial\_pose\_a} \\
    double型,default: 0.0 \\
    初期姿勢の向きの平均値です.正規分布とともにフィルタの初期化に使用されます.
  \item \textsf{initial\_cov\_xx} \\
    double型,default: 0.0 \\
    初期姿勢の$x$座標の共分散です.正規分布とともにフィルタの初期化に使用されます.
  \item \textsf{initial\_cov\_yy} \\
    double型,default: 0.0 \\
    初期姿勢の$y$座標の共分散です.正規分布とともにフィルタの初期化に使用されます.
  \item \textsf{initial\_cov\_aa} \\
    double型,default: 0.0 \\
    初期姿勢の向きの共分散です.正規分布とともにフィルタの初期化に使用されます.
  \item \textsf{gui\_publish\_rate} \\
    double型,default: -1.0 \\
    レーザースキャンと軌跡を可視化する際の最大配信周期です.負の値を設定するとデータの配信が無効化されます.
  \item \textsf{save\_pose\_rate} \\
    double型,default: 0.5 \\
    最後に推定された姿勢を\textsf{initial\_pose\_*}と\textsf{initial\_cov\_*}としてパラメータサーバーに保存する頻度の最大値です.保存された姿勢は,後にもう一度\textsf{amcl}を実行する際にフィルタを初期化するために使用されます.
  \item \textsf{use\_map\_topic} \\
    bool型,default: \textsf{false} \\
    このパラメータを\textsf{true}に設定すると,\textsf{amcl}は地図を受信するためにサービスコールを行うのではなく,トピックを購読するようになります.
  \item \textsf{first\_map\_only} \\
    bool型,default: \textsf{false} \\
    このパラメータを\textsf{true}に設定すると,\textsf{amcl}は新しい地図を受信するたびに地図を更新するのではなく,最初に購読した地図のみを使用するようになります.
  \item \textsf{selective\_resampling} \\
    bool型,default: \textsf{false} \\
    このパラメータを\textsf{true}に設定すると,GMappingで採用されている$N_{\text{eff}}$の値を基準としたリサンプリング処理が行われるようになります.
\end{itemize}

\subsubsection{レーザーモデルに関するパラメータ}

\textsf{amcl}ではレーザースキャンの観測モデルとして,``beam model''と``likelihood-field model''の2つをサポートしています.
各パラメータは各モデルの確率分布の重み付け係数と形状パラメータです.

``beam model''では\textsf{z\_hit},\textsf{z\_short},\textsf{z\_max},\textsf{z\_rand}の4つが使用されます.
``likelihood-field model''では\textsf{z\_hit}と\textsf{z\_rand}の2つのみが使われます.
どちらのモデルを使う場合でも,各重み係数は和が1.0になるようにしなければなりません.

\begin{itemize}
  \item \textsf{laser\_min\_range} \\
    double型,default: -1.0 \\
    レーザーの最小距離を指定します.負の値が設定された場合,\textsf{sensor\_msgs/LaserScan}メッセージに含まれている値を使用します.
  \item \textsf{laser\_max\_range} \\
    double型,default: -1.0 \\
    レーザーの最大距離を指定します.負の値が設定された場合,\textsf{sensor\_msgs/LaserScan}メッセージに含まれている値を使用します.
  \item \textsf{laser\_max\_beams} \\
    int型,default: 30 \\
    1スキャン当たりに使用するレーザービームの数を指定します.デフォルトの設定だと,1スキャンのうち等間隔に30個のビームだけ抜き出してフィルタ処理に使用することになります.
  \item \textsf{laser\_z\_hit} \\
    double型,default: 0.95 \\
    観測モデルのうち,正規分布成分の重み係数です.
  \item \textsf{laser\_z\_short} \\
    double型,default: 0.1 \\
    観測モデルのうち,指数分布成分の重み係数です.
  \item \textsf{laser\_z\_max} \\
    double型,default: 0.05
    観測モデルのうち,無限値観測に起因する一様分布成分の重み係数です.
  \item \textsf{laser\_z\_rand} \\
    double型,default: 0.05 \\
    観測モデルのうち,ランダム観測に起因する一様分布成分の重み係数です.
  \item \textsf{laser\_sigma\_hit} \\
    double型,default: 0.2 \\
    正規分布に従うノイズの標準偏差です.単位はメートルです.
  \item \textsf{laser\_lambda\_short} \\
    double型,default: 0.1 \\
    指数分布に従うノイズのパラメータです.
  \item \textsf{laser\_likelihood\_max\_dist} \\
    double型,default: 2.0 \\
    尤度場モデルにおいて,地図上の障害物からどれだけの距離を膨張させるかを指定します.
  \item \textsf{laser\_model\_type} \\
    string型,default: ``likelihood\_field''  \\
    観測モデルの種類を指定します.\textsf{beam},\textsf{likelihood\_field},\textsf{likelihood\_field\_prob}の3つから指定できます.
    \textsf{likelihood\_field\_prob}は\textsf{likelihood\_field}とほぼ同じですが,有効化されている場合はビームスキップ処理を行います.
  \item \textsf{do\_beamskip} \\
    bool型,default: false \\
    ビームスキップ処理を行うかどうかを指定します.
  \item \textsf{beam\_skip\_distance} \\
    double型,default: 0.5 \\
    ビームが地図とマッチしているかどうかを判断するための閾値です.
    ビーム終点と一番近い障害物セルとの距離がこの閾値よりも大きいとき,そのビームは地図とマッチしないと判断されます.
  \item \textsf{beam\_skip\_threshold} \\
    double型,default: 0.3 \\
    ビームがスキップされる基準の閾値です.ビームスキップ処理では,あるビームに対して,ビームが地図とマッチしないパーティクルが十分存在するときビームがスキップされます.
    全てのパーティクルに対し,ビームがマッチしないパーティクルが一定の割合以上のとき,ビームスキップが行われます.
    その割合を決定するのがこのパラメータです.
  \item \textsf{beam\_skip\_error\_threshold} \\
    double型,default: 0.9 \\
    あまりにも多くのビームがスキップされた場合,\textsf{amcl}は逆に全てのビームを処理するようになります.
    スキップ対象のビームの数が,フィルタ更新に使用する全てのビームの数(\textsf{laser\_max\_beams}で指定した数)に対して一定割合以上になったときにビームスキップ処理を破棄します.
    その割合を決定するのがこのパラメータです.
\end{itemize}

\subsubsection{オドメトリモデルに関するパラメータ}

\begin{itemize}
  \item \textsf{odom\_model\_type} \\
    string型,default: ``diff'' \\
    オドメトリモデルで使用するロボットの移動機構の種別を指定します.
    指定できるタイプは``diff'',``omni'',``diff-corrected'',``omni-corrected''の4つです.
    ``corrected''が付いたモデルはバグ修正がされたもので,古いタイプは互換性のために残されています.
    従って,新規で作る場合は修正されたモデルを使用するべきでしょう.
    ただし,ROS Wikiに記述されているように,オドメトリモデルの各係数のデフォルト値は古いモデルにのみフィットする値であり,修正されたモデルではもっと小さい値を使うべきとされています.
    パラメータチューニングを行う際は,ロボットを実際に走らせ,オドメトリの誤差がどれくらい生じるかをもとにして行わなければなりません.
  \item \textsf{odom\_alpha1} \\
    double型,default: 0.2 \\
    ロボットの回転動作によって,オドメトリの回転成分にどれくらいのノイズが発生するかを指定します.
  \item \textsf{odom\_alpha2} \\
    double型,default: 0.2 \\
    ロボットの直進動作によって,オドメトリの回転成分にどれくらいのノイズが発生するかを指定します.
  \item \textsf{odom\_alpha3} \\
    double型,default: 0.2 \\
    ロボットの直進動作によって,オドメトリの直進成分にどれくらいのノイズが発生するかを指定します.
  \item \textsf{odom\_alpha4} \\
    double型,default: 0.2 \\
    ロボットの回転動作によって,オドメトリの直進成分にどれくらいのノイズが発生するかを指定します.
  \item \textsf{odom\_alpha5} \\
    double型,default: 0.2 \\
    ロボットの並進に関係するノイズパラメータです.``omni''モデルでのみ使用されます.
  \item \textsf{odom\_frame\_id} \\
    string型,default: ``odom'' \\
    オドメトリ座標系のフレーム名を指定します.
  \item \textsf{base\_frame\_id} \\
    string型,default: ``base\_link'' \\
    ロボットのベースリンクの名前を指定します.
  \item \textsf{global\_frame\_id} \\
    string型,default: ``map'' \\
    地図座標系のフレーム名を指定します.
\end{itemize}

\subsection{KLDサンプリング}

\textsf{amcl}はKLDサンプリング\cite{amcl}というアルゴリズムを用いることで,パーティクルの数を可変にしています.
KLDサンプリングは,ロボットの姿勢の真の信念分布とパーティクルによって近似した分布との誤差を数値化し,誤差が閾値$\varepsilon$を超えてしまう確率が$\delta$以内に収まるように,パーティクルの数を決定します.

真の信念分布とパーティクルによる分布との誤差を数値化するために,カルバック・ライブラー情報量(Kullback-Leibler divergence)というものが使用されます.
KLDサンプリングの具体的なアルゴリズムの導出については,筆者の理解が追いついていないのでここでは説明できません.
元となった論文を読むか,詳解確率ロボティクス\cite{上田2019}第7章に詳細が記述されているので,そちらを参考にして下さい.

まず,カルバック・ライブラー情報量(KL情報量)の概要について(筆者が理解できた範囲で)説明します.
KL情報量は,状態空間$\chi$中の2つの確率密度関数$p$,$q$に対して,

\begin{equation}
  \begin{split}
    D_{\text{KL}}(p \mid\mid q) &= \int_{\bm{x} \in \chi} p(\bm{x}) \log{\frac{p(\bm{x})}{q(\bm{x})}}d\bm{x} \\
    &= \langle \log{\frac{p(\bm{x})}{q(\bm{x})}} \rangle_{p(\bm{x})} \\
    &= \langle \log{p(\bm{x})} \rangle_{p(\bm{x})} - \langle \log{q(\bm{x})} \rangle_{p(\bm{x})}
  \end{split}
\end{equation}

\noindent
と定義されます.
ただし,$\langle z \rangle_{p(z)}$は確率分布$p(z)$に従う確率変数$z$の期待値で,$E_{p(z)}[z]$とも表記されます.
離散的な状態空間$M$中の確率分布$P$,$Q$の場合でも同様に,

\begin{equation}
  D_{\text{KL}}(P \mid\mid Q) = \sum_{s \in M} P(s)\log{\frac{P(s)}{Q(s)}} = \langle \log{P(s)} \rangle_{P(s)} - \langle \log{Q(s)} \rangle_{P(s)}
\end{equation}

\noindent
と定義されます.
ただし,$s$は離散的な状態空間の中の一区画で,ビンと呼ばれるものです.

2つの確率分布$p$,$q$もしくは$P$,$Q$が一致するとき,対数の中の分数が1になるため,KL情報量は0になります.
それ以外の場合は,KL情報量は常に正の値を取ります.

Adaptive MCLでは状態空間を離散化して考えます.
2次元平面を移動するロボットが取る姿勢は$\bm{x} = (x, y, \theta)^T$の3次元なので,状態空間は3次元になります.
それぞれの軸を等間隔に区切ると,状態空間は網目になります.
そのうちの1区画を$s$とし,それぞれの区画を区別するため,$s_0, s_1, \ldots, s_{k-1}$のようにインデックスを付与します.
ビンの数は$k$個になります.

そして,各ビン$s_j$内に真の姿勢$\bm{x}^*$が存在する確率を考えます.
この確率は,真の分布$b^{*}$から考えられる$P^{*}(s)$と,パーティクルによって表現された分布$\hat{P}(s)$の2種類が存在します.
ここで,$N$個ある内の$i$個目のパーティクルは$\xi^{(i)} = (\bm{x}^{(i)}, w^{(i)})$で表現されます.
$P^{*}(s)$は式\ref{eq:true_bin_probability}で,$\hat{P}(s)$は式\ref{eq:particle_bin_probability}で表されます.

\begin{equation}
  P^{*}(\bm{x}^{*} \in s_{j}) = \int_{\bm{x} \in s_{j}} b^{*}(\bm{x})d\bm{x}
  \label{eq:true_bin_probability}
\end{equation}

\begin{equation}
  \hat{P}(\bm{x}^{*} \in s_{j}) = \sum_{\bm{x}^{(i)} \in s_{j}} w^{(i)}
  \label{eq:particle_bin_probability}
\end{equation}

$\hat{P}(s)$は特に,リサンプリング後で全てのパーティクルの重みが等しいとき,

\begin{equation}
  \hat{P}(\bm{x}^{*} \in s_{j}) = \sum_{\bm{x}^{(i)} \in s_{j}} \frac{1}{N} = \frac{n_j}{N}
  \label{eq:particles_with_equal_weights}
\end{equation}

\noindent
となります.
ここで,$n_j$はビン$s_j$内にあるパーティクルの数です.

尚,真の分布$P^{*}(s)$は未知であり,計算することはできません.
MCLで求めるのは$\hat{P}(s)$の方であり,これを使って真の分布を近似するのがMCLであると言えます.

ここで,2つの分布$P^{*}(s)$と$\hat{P}(s)$のKL情報量が誤差の閾値$\varepsilon$以内に収まる確率

\begin{equation}
  P[D_{\text{KL}}(\hat{P} \mid\mid P^{*}) \leq \varepsilon]
\end{equation}

\noindent
を考えます.
真の分布$P^{*}(s)$は未知ですが,この式は$N$個のパーティクルがどれだけ真の分布を近似できているかを数値化したものになります.
$N$が大きくなれば,$\hat{P}$は$P^{*}$に近づきます.

$P^{*}$の各ビンの確率を$p^{*}_{0}, p^{*}_{1}, p^{*}_{2}, \ldots p^{*}_{k-1}$と,$\hat{P}$の各ビンの確率を$\hat{p}_{0}, \hat{p}_{1}, \hat{p}_{2}, \ldots \hat{p}_{k-1}$と書くと,KL情報量$D_{\text{KL}}(\hat{P} \mid\mid P^{*})$は

\begin{equation}
  D_{\text{KL}}(\hat{P} \mid\mid P^{*}) = \sum^{k-1}_{j=0} \hat{p}_{j}\log{\frac{\hat{p}_{j}}{p^{*}_j}}
\end{equation}

\noindent
という式になります.繰り返しますが,$p^{*}_{0}, p^{*}_{1}, p^{*}_{2}, \ldots p^{*}_{k-1}$は未知です.

更に,閾値$\delta$を考え,

\begin{equation}
  P[D_{\text{KL}}(\hat{P} \mid\mid P^{*}) \leq \varepsilon] \geq 1 - \delta
\end{equation}

\noindent
を満たすには,パーティクル数$N$がいくつ必要なのか,という問題を考えます.
閾値$\delta$は,「真の分布とパーティクルの分布とのKL情報量が$\varepsilon$以内に収まらない確率」です.

ここで,Adaptive MCLの問題である「真の分布を近似するために$N$個のパーティクルを使う」ということを,「式\ref{eq:true_bin_probability}に従う確率でビンを選んで,ビン内のどこかにパーティクルを置く」という作業と解釈します.
そして,この作業の結果を式\ref{eq:particles_with_equal_weights}が表していると解釈します.

このとき,「$P^{*}$に従って$N$回ビンを選び,その結果,各ビンがそれぞれ$n_0, n_1, n_2, \ldots , n_{k-1}$回選ばれる」という事象の確率は,多項分布に従います.

\begin{equation}
  \begin{split}
    P^{*}_{M}(n_0, n_1, n_2, \ldots , n_{k-1} \mid p^{*}_{0}, p^{*}_{1}, p^{*}_{2}, \ldots , p^{*}_{k-1}) &= \frac{N!}{n_{0}!n_{1}!n_{2}!\ldots n_{k-1}}(p^{*}_{0})^{n_{0}}(p^{*}_{1})^{n_{1}}(p^{*}_{2})^{n_{2}} \ldots (p^{*}_{k-1})^{n_{k-1}} \\
    &= \frac{N!}{\prod^{k-1}_{j=0}}\prod^{k-1}_{j=0}(p^{*}_{j})^{n_{j}}
  \end{split}
  \label{eq:multinomial_belief_distribution_1}
\end{equation}

式\ref{eq:multinomial_belief_distribution_1}は,真の分布からパーティクルをサンプリングしたら,どのような分布がどれだけの確率で現れるかを表す確率分布です.
$N$が大きいほど真の分布$P^{*}$に近い分布が高い確率で得られるようになります.
$P^{*}_{M}$はその性質を表す分布です.
各ビンの選ばれる回数と,各ビンにおける真の確率を,それぞれ$\bm{n} = \{n_{0}, n_{1}, n_{2} \ldots , n_{k-1}\}$,$\Theta = \{p^{*}_{0}, p^{*}_{1}, p^{*}_{2}, \ldots , p^{*}_{k-1}\}$とリストにまとめ,$P^{*}_{M}(\bm{n} \mid \Theta^{*})$と表すと,

\begin{equation}
  P^{*}_{M}(\bm{n} \mid \Theta^{*}) = \frac{N!}{\prod^{k-1}_{j=0}}\prod^{k-1}_{j=0}(p^{*}_{j})^{n_{j}}
  \label{eq:multinominal_belief_distribution_2}
\end{equation}

\noindent
となります.

繰り返しますが,真の確率のリスト$\Theta = \{p^{*}_{0}, p^{*}_{1}, p^{*}_{2}, \ldots , p^{*}_{k-1}\}$は未知なので,式\ref{eq:multinomial_belief_distribution_2}は計算できません.
そこで,式\ref{eq:multinominal_belief_distribution_2}の値を,実際にサンプリングして集計した回数$\bm{n} = \{n_{0}, n_{1}, n_{2} \ldots , n_{k-1}\}$と式\ref{eq:particles_with_equal_weights}で近似することを考えてみます.
ビン$s_{j}$が選ばれる確率を$\hat{p}_{j}$としたとき,式\ref{eq:particles_with_equal_weights}から,$\hat{p}_{j}=\frac{n_{j}}{N}$となります.
この確率のリストを$\hat{\Theta} = \{\hat{p}_{0}, \hat{p}_{1}, \hat{p}_{2}, \ldots , \hat{p}_{k-1}\}$としたとき,$\hat{\Theta}$に従って再度サンプリングすると,また回数が$\bm{n}$となる確率は,

\begin{equation}
  \hat{P}_{M}(\bm{n} \mid \hat{\Theta}) = \frac{N!}{\prod^{k-1}_{j=0}(n_{j}!)}\prod^{k-1}_{j=0}(\hat{p}_{j})^{n_{j}}
\end{equation}

\noindent
となります.
ここで,$\Theta = \{p_{0}, p_{1}, p_{2}, \ldots , p_{k-1}\}$を変数とする尤度関数を考えます.

\begin{equation}
  L(\Theta \mid \bm{n}) = P_{M}(\bm{n} \mid \Theta) = \frac{N!}{\prod^{k-1}_{j=0}}\prod^{k-1}_{j=0}(p_{j})^{n_{j}}
\end{equation}

そして,$\hat{P}_{M}$の$\hat{\Theta}$,$P^{*}_{M}$の$\Theta^{*}$に対応する対数尤度比を考えます.

\begin{equation}
  \begin{split}
    \log{\lambda_{N}} &= \log{\frac{L(\hat{\Theta} \mid \bm{n})}{L(\Theta^{*} \mid \bm{n})}} = \log{\frac{\prod^{k-1}_{j=0}\hat{p}_{j}^{n_{j}}}{\prod^{k-1}_{j=0}{p^{*}_{j}}^{n_{j}}}} = \log{\prod^{k-1}_{j=0}\left(\frac{\hat{p}_{j}}{p^{*}_{j}}\right)^{n_{j}}}\\
    &= \sum^{k-1}_{j=0} \log{\left(\frac{\hat{p}_{j}}{p^{*}_{j}}\right)^{n_{j}}} = \sum^{k-1}_{j=0} n_{j}\log{\left(\frac{\hat{p}_{j}}{p^{*}_{j}}\right)}
  \end{split}
\end{equation}

この対数尤度比は,「$\bm{n}$が得られたときに$\hat{\Theta}$が$\Theta^{*}$の近似として適切か否か」の指標になります.
ただし,真の確率$p^{*}_{j}$があるので計算自体は不可能です.
更に,$n_{j} = \hat{p}_{j}N$なので,

\begin{equation}
  \log{\lambda_{N}} = \sum^{k-1}_{j=0} \hat{p}_{j}N\log{\left(\frac{\hat{p}_{j}}{p^{*}_{j}}\right)} = N\sum^{k-1}_{j=0} \hat{p}_{j}\log{\left(\frac{\hat{p}_{j}}{p^{*}_{j}}\right)} = N \cdot D_{\text{KL}}(\hat{P} \mid \mid P^{*})
  \label{eq:log_likelihood_ratio}
\end{equation}

\noindent
となります.式変形の最後にKL情報量が現れます.

何故このような回りくどい式変形をしているかというと,$\hat{\Theta}$を使ってビンからパーティクルを$N$個選ぶということを繰り返すと,$N$が大きい場合に,$\log{\lambda_{N}}$の値を2倍した値の分布が,自由度$k-1$のカイ二乗分布に従う性質を利用するためです.
つまり,真の確率$p^{*}_{j}$や対数尤度比$\log{\lambda_{N}}$の値がわからなくても,$\log{\lambda_{N}}$の統計的性質はわかる,ということです.

カイ二乗分布とは,$a \sim \mathcal{N}(0, 1)$のときに$a$を$k$回ドローした値$a_{1}, a_{2}, \ldots , a_{k}$の二乗の和

\begin{equation}
  x = a_{1}^{2} + a_{2}^{2} + \cdots + a_{k}^{2} 
\end{equation}

\noindent
が従う分布で,

\begin{equation}
    \chi_{k}^{2}(x) = \frac{1}{2^{\frac{k}{2}} \Gamma \left(\frac{k}{2}\right)} x^{\frac{k}{2} - 1}e^{-\frac{x}{2}}
\end{equation}

\begin{equation}
  \Gamma (\alpha) = \int^{\infty}_{0} t^{\alpha - 1} e^{-t} dt
\end{equation}

\noindent
という形をしています.
$\Gamma (\alpha)$はガンマ関数と呼ばれる特殊な関数です.

この性質を用いると,$N$が有限なのであくまで近似ですが,

\begin{equation}
  2 \log{\lambda_{N}} \sim \chi_{k-1}^{2}
  \label{eq:chi_approximation}
\end{equation}

\noindent
となります.

今度は,式\ref{eq:log_likelihood_ratio}を用いて,式\ref{eq:particles_with_equal_weights}を次のように変形します.

\begin{equation}
  \begin{split}
    P[D_{\text{KL}}(\hat{P} \mid \mid P^{*}) \leq \varepsilon] &\geq 1 - \delta \\
    p[2\log{\lambda_{N}} \leq 2N\varepsilon] &\geq 1 - \delta
  \end{split}
\end{equation}

これは,「$2\log{\lambda_{N}}$の値が$2N\varepsilon$を下回る確率が$1 - \delta$になるように,$N$を選ぶ」という問題になります.
この問題は,

\begin{equation}
  \int^{2\log{\lambda_{N}}}_{0} \chi^{2}_{k-1}(x) dx = 1 - \delta
\end{equation}

\noindent
を満たす$2\log{\lambda_{N}}$ -- つまり自由度$k-1$のカイ二乗分布の分位数 -- を求め,この分位数を$y$として,

\begin{equation}
  \begin{split}
    2N\varepsilon &> y \\
    N &> \frac{y}{2\varepsilon}
  \end{split}
\end{equation}

\noindent
となるように$N$を選ぶことで解けます.
これを満たす最小の$N$が,真の分布を表現するために必要なパーティクルの数,ということになります.
言い換えれば,「自由度$k-1$のカイ二乗分布($k$はビンの数)の分位数$y$」と,「KL情報量の誤差$\varepsilon$」を決めれば,分布の表現に必要なパーティクル数$N$を決定することができる,ということです.

ところで,自由度$k-1$のカイ二乗分布の分位数$y$はどのように求めればよいのでしょうか?
統計学のライブラリが充実している言語(例えばPythonとか)でMCLを実装する場合にはそれを使えばいいのでしょうが,そのようなライブラリが無かったり,計算リソースがシビアな環境のときは困ったことになります.
このような問題を解決するには,ウィルソン-フィルファーティ変換(Wilson-Hilferty Transform)という方法を使います.
これは,カイ二乗分布の分位数を,標準正規分布の分位数を含んだ式で近似することができるもので,以下のような式で表されます.

\begin{equation}
  y \sim (k-1) \left( 1 - \frac{2}{9(k-1)} + \sqrt{\frac{2}{9(k-1)}}z_{1-\delta} \right) ^{3}
\end{equation}

$z_{1-\delta}$は標準正規分布$\mathcal{N}(0, 1)$の分位数で,標準正規分布を$-\infty$から$z_{1-\delta}$まで積分したときの確率が$1-\delta$になるような値です.
これを用いることで,パーティクル数$N$は,

\begin{equation}
  N > \frac{y}{2\varepsilon} \sim \frac{k-1}{2\varepsilon} \left( 1 - \frac{2}{9(k-1)} + \sqrt{\frac{2}{9(k-1)}}z_{1-\delta} \right)^{3}
\end{equation}

\noindent
と計算することができます.
ROSの\textsf{amcl}でもこの変換が用いられています\footnote{\url{https://github.com/ros-planning/navigation/blob/7f22997e6804d9d7249b8a1d789bf27343b26f75/amcl/src/amcl/pf/pf.c\#L535-L540}}.
ウィルソン-フィルファーティ変換により,「標準正規分布の分位数$z_{1-\delta}$」と「KL情報量の誤差$\varepsilon$」を決めれば,必要なパーティクル数$N$を決められる,というわけです.
\textsf{amcl}のノードにおける\textsf{kld\_z}が標準正規分布の分位数,\textsf{kld\_err}がKL情報量の誤差に当たるので,これら2つのパラメータを設定することでKLDサンプリングの挙動をコントロールできるということになります.

あと計算に必要なのはビン数$k$ですが,\textsf{amcl}ではKD-Treeを使ってビン数を決定しているようです.
ビン数の計算に関しては筆者がまだ理解できていないので,ここで言及するのは控えます.

\subsection{Augmented MCL}

次に,MCLでロボットの誘拐問題に対処するためのAugmented MCLについて簡単に解説します.
ロボットの誘拐問題とは,「今まで自己位置を推定できていたロボットが突然別の場所に連れ去られる」問題のことです.
MCLでは今までロボットが持っていた信念分布を元に現在時刻の位置を推定するため,誘拐問題に対処するにはそれまでロボットが持っていた信念分布を一旦否定しなければなりません.

これに対処するためには,観測によって得られた情報を元に信念分布をリセットすればよいことになります.
その際,リセットをするかどうかを判断する指標として使うのが,周辺尤度$\alpha$です.
周辺尤度$\alpha$は信念分布に対して観測がどれだけあり得ないかを数値化したもので,各パーティクルの重みの総和です.

\begin{equation}
  \alpha = \sum_{i=0}^{N-1} w^{(i)}
\end{equation}

ある時刻で得られた観測と信念分布が乖離したとき,$\alpha$の値は小さくなります.
従って,ある閾値を設け,$\alpha$の値がそれを下回ったら信念分布をリセットすればよいということになります.

しかし,単純に$\alpha$の値を監視するだけでは,思いがけないタイミングでリセットが発生してしまう可能性があります.
センサに大きなノイズが混入したり,誤った観測が得られてしまったとき等にそれが起こりえます.

これを防ぐためには,$\alpha$の値をある程度長い時間に渡って監視し,$\alpha$の値の変化を見てリセットすべきかを判断します.
Augmented MCLでは,$\alpha_{\text{slow}}$と$\alpha_{\text{fast}}$の2つの値を計算します.
この値は,センサ値が得られる度に,式\ref{eq:augmented_mcl_alpha_update}に従って更新します.

\begin{equation}
  \begin{split}
    \alpha_{\text{slow}} &\leftarrow \alpha_{\text{slow}} + \alpha_{\text{th-slow}}(\alpha - \alpha_{\text{slow}}) \\
    \alpha_{\text{fast}} &\leftarrow \alpha_{\text{fast}} + \alpha_{\text{th-fast}}(\alpha - \alpha_{\text{fast}}) \\
  \end{split}
  \label{eq:augmented_mcl_alpha_update}
\end{equation}

$\alpha_{\text{th-slow}}$と$\alpha_{\text{th-fast}}$はそれぞれ閾値で,$0 < \alpha_{\text{th-slow}} \ll \alpha_{\text{th-fast}} < 1$と設定します.
このようにすると,周辺尤度$\alpha$の値が連続で小さくなっていったとき,$\alpha_{\text{fast}}$が先に小さくなっていき,遅れて$\alpha_{\text{slow}}$が0に近づいていきます.
この$\alpha_{\text{slow}}$と$\alpha_{\text{fast}}$を用いて,現在$N$個あるパーティクルのうち,次の式で計算される$\tilde{N}$個のパーティクルだけリセット対象とします.

\begin{equation}
  \tilde{N} = N \cdot \max \left(0, 1 - \frac{\alpha_{\text{fast}}}{\alpha_{\text{slow}}}\right)
\end{equation}

$\alpha$が連続で小さくなったとき,$\frac{\alpha_{\text{fast}}}{\alpha_{\text{slow}}}$が次第に小さくなっていくので,$\tilde{N}$が大きくなっていきます.
閾値$\alpha_{\text{th-fast}}$はだいたい0.1程度の大きさに,$\alpha_{\text{th-slow}}$は0.001程度にするのが一般的です.

\subsection{レーザースキャンのビームモデル}

\subsection{レーザースキャンの尤度場モデル}

\subsection{ビームスキップ処理}

\subsection{パラメータ調整の勘所}

\end{document} 