\documentclass[{../../master}]{subfiles}
\graphicspath{{../../}}  % 個別コンパイル時の画像パスを解決する

\begin{document}

\section{パラメータファイルの記述}

\subsection{ディレクトリ構造}

ここからはパラメータファイルを記述していきます.
Navigation Stackはいくつものノードが集まって自律走行を実現しているため,パラメータの数はこれまでと比べて膨大になります.
従って,1つのファイルにパラメータを全て列挙するのではなく,ある程度ファイルやディレクトリを分割して記述することとします.

具体的には,\textsf{adamr2\_navigation}パッケージの\textsf{config/}ディレクトリを\ref{code:directory_structure_of_adamr2_navigation_config}のようにし,1つのノードに対して1つのファイルを割り当てるようにします.

\begin{lstlisting}[caption=Directory Structure of \textsf{config/}, label=code:directory_structure_of_adamr2_navigation_config]
config/
  ├ amcl/
  │  └ amcl.yml
  ├ costmap/
  │  ├ global_costmap.yml
  │  └ local_costmap.yml
  ├ planner/
  │  ├ global/
  │  │  └ global_planner.yml
  │  └ local/
  │     └ trajectory_planner.yml
  └ move_base.yml
\end{lstlisting}

\subsection{\textsf{amcl}のためのパラメータの記述}

まずは自己位置推定パッケージ\textsf{amcl}のためのパラメータを記述していきます.
\textsf{config/}ディレクトリ以下に\textsf{amcl}という名前のディレクトリを作り,その中に\textsf{amcl.yml}という名前のファイルを作成します.
そして,コード\ref{code:amcl_yml}のように記述します.

\begin{lstlisting}[language=YAML, label=code:amcl_yml, caption=\textsf{amcl.yml}]
min_particles: 30   # default = 100
max_particles: 1000 # default = 5000

kld_err: 0.01 # default = 0.01
kld_z: 0.99   # default = 0.99

update_min_d: 0.05 # default = 0.2
update_min_a: 0.05 # default = PI / 6.0 ~= 0.5

resample_interval: 1      #  default = 2
transform_tolerance: 0.3  # default = 0.1
recovery_alpha_slow: 0.0  # default = 0.0
recovery_alpha_fast: 0.0  # default = 0.0

gui_publish_rate: 10.0  # default = -1.0[Hz](disable)

save_pose_rate: 0.5     # defautl = 0.5
use_map_topic: true     # defaut = false
first_map_only: false   # default = false

laser_min_range: -1.0               # default = -1.0
laser_max_range: -1.0               # default = -1.0
laser_max_beams: 60                 # default = 30
laser_z_hit: 0.95                   # default = 0.95
#laser_z_short: 0.05                # default = 0.1
#laser_z_max: 0.05                  # default = 0.05
laser_z_rand: 0.05                  # default = 0.05
laser_sigma_hit: 0.02               # default = 0.2
#laser_lambda_short: 0.1            # default = 0.1
laser_likelihood_max_dist: 2.0      # default = 2.0
laser_model_type: likelihood_field  # default = likelihood_field

odom_model_type: diff-corrected # default = diff
odom_alpha1: 0.2                # default = 0.2
odom_alpha2: 0.2                # default = 0.2
odom_alpha3: 0.2                # default = 0.2
odom_alpha4: 0.2                # default = 0.2

odom_frame_id: odom             # default = odom
base_frame_id: base_footprint   # default = base_link
global_frame_id: map            # default = map
tf_broadcast: true              # default = true
\end{lstlisting}

ほとんどの項目はデフォルト値のままにしています.
SLAMのときと同じく,\textsf{base\_frame\_id}はロボットのURDF構造に合わせて書き換えましょう.

\subsection{\textsf{costmap\_2d}のためのパラメータの記述}

次にコストマップのパラメータを設定していきます.
コストマップとはロボットの自律走行の際に用いられる,障害物の存在確率を示す占有格子地図の事です.
単に障害物の情報を表現するだけでなく,ロボットの輪郭形状を考慮してロボットが移動可能な領域を示す役割も果たします.
コストマップはNavigation Stackの3つの主要機能(自己位置推定,経路計画,コストマップ)の内の1つであり,これの設定を適切に行わなければ,うまくナビゲーションを行うことができません.

Navigation Stackにおいては,\textsf{costmap\_2d}というパッケージがコストマップを提供します.
パッケージは\textsf{move\_base}からプラグインとして利用されます.
\textsf{move\_base}はグローバルとローカルの2つのコストマップを持ち,各々のコストマップのパラメータはそれぞれ以下のような名前空間で区切られて管理されます.

\begin{itemize}
  \item グローバルコストマップ:\textsf{move\_base/global\_costmap/**}
  \item ローカルコストマップ:\textsf{move\_base/local\_costmap/**}
\end{itemize}

グローバルとローカルのコストマップが別々のインスタンスとして\textsf{move\_base}に使用されますが,元となっているクラスは同じものです.
パラメータの設定によってローカルコストマップのように振る舞ったり,グローバルコストマップのように振る舞ったりするだけです.\cite{costmap_2d_configuration}

\textsf{costmap\_2d}のパラメータ設定の仕様はHydroから大きく変わっており,Hydroより以前のパラメータ設定の構文はPre-Hydro形式と呼ばれています.
多くのネット記事や日本語のROS書籍ではPre-Hydro形式でコストマップの設定を行っているのですが,やはり最新の形式で設定するのが良いでしょう.

まずグローバルコストマップとローカルコストマップのそれぞれのパラメータを記述するファイルを作ります.
コード\ref{code:directory_structure_of_adamr2_navigation_config}を参考に,\textsf{global\_costmap.yml}と\textsf{local\_costmap.yml}という名前のファイルを作成します.
そして,コード\ref{code:global_costmap_yml},コード\ref{code:local_costmap_yml}のようにそれぞれ記述します.

\begin{lstlisting}[language=YAML, label=code:global_costmap_yml, caption=\textsf{global\_costmap.yml}]
global_costmap:
  # Plugins
  plugins:
    - {name: static_layer, type: "costmap_2d::StaticLayer"}
    - {name: obstacle_layer, type: "costmap_2d::ObstacleLayer"}
    - {name: inflation_layer, type: "costmap_2d::InflationLayer"}
  
  # Coordinate frame and tf parameters
  global_frame: map
  robot_base_frame: base_footprint
  transform_tolerance: 0.2  # default: 0.2
  footprint: [[0.35, 0.225], [0.35, -0.225], [-0.37, -0.225], [-0.37, 0.225]]

  # Rate parameters
  update_frequency: 2.0     # default: 5.0
  publish_frequency: 2.0    # default: 0.0(disable)

  # Map management parameters
  rolling_window: false           # default: false
  always_send_full_costmap: false # default:false

  # parameters of each plugins
  static_layer:
    unknown_cost_value: -1      # default: -1
    lethal_cost_threshold: 100  # default: 100
    map_topic: map              # default: map
    first_map_only: false       # default: false
    subscribe_to_updates: false # default: false
    track_unknown_space: true   # default: true
    use_maximum: false          # default: false
    trinary_costmap: true       # default: true

  obstacle_layer:
    observation_sources: laser_scan
    laser_scan:
      topic: /adamr2/scan
      sensor_frame: lidar_link
      observation_persistence: 0.0  # default: 0.0
      expected_update_rate: 0.2     # default: 0.0
      data_type: LaserScan
      clearing: true        # default: false
      marking: true         # default: true
      obstacle_range: 2.5   # default: 2.5
      raytrace_range: 3.0   # default: 3.0
      inf_is_valid: true   # default: false

  inflation_layer:
    inflation_radius: 0.5      # default: 0.55
    cost_scaling_factor: 3.66   # default: 10.0
\end{lstlisting}

\begin{lstlisting}[language=YAML, label=code:local_costmap_yml, caption=\textsf{local\_costmap.yml}]
local_costmap:
  # Plugins
  plugins:
    - {name: obstacle_layer, type: "costmap_2d::ObstacleLayer"}
    - {name: inflation_layer, type: "costmap_2d::InflationLayer"}

  # Coordinate frame and tf parameters
  global_frame: odom
  robot_base_frame: base_footprint
  transform_tolerance: 0.2  # default: 0.2
  footprint: [[0.35, 0.225], [0.35, -0.225], [-0.37, -0.225], [-0.37, 0.225]]

  #  Rate parameters
  update_frequency: 5.0    # default: 5.0
  publish_frequency: 5.0   # default: 0.0

  # Map management parameters
  rolling_window: true
  always_send_full_costmap: false
  width: 8.0        # default: 10.0
  height: 8.0       # default: 10.0
  resolution: 0.05  # default: 0.05

  # parameters of each plugins
  obstacle_layer:
    observation_sources: laser_scan pointcloud_sensor
    laser_scan:
      topic: /adamr2/scan
      sensor_frame: lidar_link
      observation_persistence: 0.0  # default: 0.0
      expected_update_rate: 0.2     # default: 0.0
      data_type: LaserScan
      clearing: true        # default: false
      marking: true         # default: true
      obstacle_range: 2.5   # default: 2.5
      raytrace_range: 3.0   # default: 3.0
      inf_is_valid: false   # default: false
    pointcloud_sensor:
      topic: /adamr2/camera/depth/color/points
      sensor_frame: camera_depth_frame
      observation_persistence: 0.0  # default: 0.0
      expected_update_rate: 0.1     # default: 0.0
      data_type: PointCloud2
      clearing: true            # default: false
      marking: true             # default: true
      max_obstacle_height: 1.0  # default: 2.0
      min_obstacle_height: 0.1  # default: 0.0
      obstacle_range: 2.5       # default: 2.5
      raytrace_range: 3.0       # default: 3.0
      inf_is_valid: false       # default: false

  inflation_layer:
    inflation_radius: 0.4     # default: 0.55
    cost_scaling_factor: 3.66 # default: 10.0
\end{lstlisting}

コストマップは各種レイヤーのプラグインを積み重ねていくことで設定します.
使用されるレイヤーは主に以下の4つです.

\begin{enumerate}
  \item \textsf{costmap\_2d::StaticLayer}:SLAMで作成した静的な地図を読み込むレイヤーです.
  \item \textsf{costmap\_2d::ObstacleLayer}:センサで観測した障害物の情報を記録するレイヤーです.
  \item \textsf{costmap\_2d::InflationLayer}:マップ上の障害物を任意の半径に膨張させるレイヤーです.
  \item \textsf{range\_sensor\_layer::RangeSensorLayer}:超音波センサ等のデータを記録するレイヤーです.
\end{enumerate}

ここでは上3つを使用します.
グローバル・ローカルで共通なのは\textsf{costmap\_2d::ObstacleLayer}と\textsf{costmap\_2d::InflationLayer}です.
\textsf{costmap\_2d::StaticLayer}はグローバルコストマップでのみ使用し,ローカルコストマップでは使用しません.
理由は,ローカルコストマップは主にロボットの周辺の動的な障害物の情報を得るために使用するものなので,静的な情報は必要としないからです.
\textsf{costmap\_2d::ObstacleLayer}は\textsf{sensor\_msgs/LaserScan}型トピックと\textsf{sensor\_msgs/PointCloud2}型トピックを受け取ることができ,それらの情報をもとに障害物の情報を表示します.
記事によってはグローバルコストマップでこのレイヤーを使用しない設定にしているものがあります.しかしその場合,人等の障害物が現れた際にグローバルパスがそれを考慮できなくなるため,ここではグローバルコストマップにも障害物レイヤーを適用しています.
\textsf{costmap\_2d::InflationLayer}はコストマップ上の障害物を表すセルを膨張させ,障害物セルの周りのセルも障害物として扱えるようにします.
これにより,ロボットを障害物に近づけさせない処理が実現できます.
ただし,この膨張範囲を大きく設定しすぎるとコストマップ全てが障害物セルで埋め尽くされてしまうため,環境に合わせて適切に設定する必要があります.

各プラグインを使用するには,\textsf{global\_costmap/plugins}パラメータに使用するプラグインを列挙しなければなりません.
その後,各プラグインに対する設定を記述します.

\subsection{\textsf{global\_planner}のためのパラメータの記述}

次にグローバルパスプランナのパラメータ設定をします.
グローバルパスプランナのパッケージは標準で\textsf{navfn}と\textsf{global\_planner}の2つがありますが,後から実装された\textsf{global\_planner}を使用するのが良いでしょう.
これらのパッケージもコストマップと同様に\textsf{move\_base}からプラグインとして使用されます.
どのパスプランナを使用するかは\textsf{/move\_base/base\_global\_planner}パラメータで指定します.

コード\ref{code:directory_structure_of_adamr2_navigation_config}を参考に,\textsf{global\_planner.yml}という名前のファイルを作成します.
そして,コード\ref{code:global_planner_yml}のように記述します.

\begin{lstlisting}[language=YAML, label=code:global_planner_yml, caption=\textsf{global\_planner.yml}]
base_global_planner: global_planner/GlobalPlanner
  GlobalPlanner:
    allow_unknown: true         # default: true
    default_tolerance: 0.3      # default: 0.0
    visualize_potential: false  # default: false
    use_dijkstra: true          # default: true
    use_quadratic: true         # default: true
    use_grid_path: false        # default: false
    old_navfn_behavior: false   # default: false
    lethal_cost: 253            # default: 253
    neutral_cost: 50            # default: 50
    cost_factor: 0.8            # default: 0.8
    publish_potential: false    # default: ture
    orientation_mode: 0         # default: 0
    orientation_window_size: 1  # default: 1
\end{lstlisting}

\textsf{global\_planner}のパラメータはデフォルト値でも十分動作します.
ユーザの環境によって調整が必要となるのは\textsf{lethal\_cost},\textsf{neutral\_cost},\textsf{cost\_factor}の3つです.
詳細なチューニングの仕方については後の節で説明します.

\subsection{\textsf{base\_local\_planner}のためのパラメータの記述}

次にローカルパスプランナのパラメータを設定します.
ローカルパスプランナも\textsf{move\_base}からプラグインとして使用され,パラメータの設定によってどのプランナを使用するかを指定できます.
Navigation Stackに標準で実装されているローカルパスプランナは以下の2つです.

\begin{enumerate}
  \item \textsf{base\_local\_planner}
  \item \textsf{dwa\_local\_planner}
\end{enumerate}

どちらもDynamic Window Approach\cite{dwa}を使用していますが,\textsf{dwa\_local\_planner}は\textsf{base\_local\_planner}の再実装版で,性能はこちらの方が上と思われます.
しかし\textsf{dwa\_local\_planner}はその場旋回を行うことがうまくできず,ロボットの向きの反対側に目的地を設定したときの挙動がちょっと怪しかったので,\textsf{base\_local\_planner}の方を使用することにしました.

これまでと同様に,コード\ref{code:directory_structure_of_adamr2_navigation_config}を参考にして\textsf{trajectory\_planner.yml}というファイルを作ります.
そして,コード\ref{code:trajectory_planner_yml}のように記述します.

\begin{lstlisting}[language=YAML, label=code:trajectory_planner_yml, caption=\textsf{trajectory\_planner.yml}]
base_local_planner: base_local_planner/TrajectoryPlannerROS
TrajectoryPlannerROS:
  # Robot Configuration Parameters
  acc_lim_x: 0.75        # default: 2.5
  acc_lim_theta: 3.14    # default: 3.2

  max_vel_x: 0.3                # default: 0.5
  min_vel_x: 0.1                # default: 0.1
  max_rotational_vel: 1.57      # default: 1.0
  min_in_place_vel_theta: 0.2   # default: 0.4
  escape_vel: -0.2              # default: -0.1

  holonomic_robot: false  # default: true

  # Goal Tolerance Parameters
  yaw_goal_tolerance: 0.1       # default: 0.05
  xy_goal_tolerance: 0.25       # default: 0.10
  latch_xy_goal_tolerance: true # default: false

  # Forward Simulation Parameters
  sim_time: 4.0                     # default: 1.0
  sim_granularity: 0.050            # default: 0.025
  angular_sim_granularity: 0.1      # default: 0.025
  vx_samples: 20                    # default: 3
  vtheta_samples: 40                # default: 20
  #controller_frequency: 10.0        # default: 20.0

  # Trajectory Scoring Parameters
  meter_scoring: true           # default: false
  path_distance_bias: 2.0       # default: 0.6
  goal_distance_bias: 0.8       # default: 0.8
  occdist_scale: 0.02           # default: 0.01
  heading_lookahead: 0.325      # default: 0.325
  heading_scoring: true         # default: false
  heading_scoring_timestep: 0.8 # default: 0.8
  dwa: true                     # default: true
  publish_cost_grid_pc: true    # default: false
  global_frame_id: odom         # default: odom

  # Oscillation Prevention Parameters
  oscillation_reset_dist: 0.1  # default: 0.05

  # Global Plan Parameters
  prune_plan: true  # default: true
\end{lstlisting}

こちらもとりあえずデフォルト値を多めに設定しています.
尚,コード\ref{code:trajectory_planner_yml}では\textsf{controller\_frequency}というパラメータをコメントアウトしていますが,これは\textsf{move\_base}本体で設定することが推奨されるパラメータだからです.
\textsf{move\_base}にも\textsf{controller\_frequency}というパラメータが存在しており,ローカルプランナ側でこのパラメータが設定されていなかった場合,\textsf{move\_base}側の値が適用されるようになっています.
制御周期は\textsf{move\_base}側で設定できた方が好ましいので,このように設定しています.

\subsection{\textsf{move\_base}のパラメータ設定}

最後に\textsf{move\_base}本体のパラメータを設定します.
\textsf{move\_base.yml}という名前のファイルを作成し,コード\ref{code:move_base_yml}のように記述します.

\begin{lstlisting}[language=YAML, label=code:move_base_yml, caption=\textsf{move\_base.yml}]
controller_frequency: 5.0  # default: 20.0
controller_patience: 15.0   # default: 15.0

planner_frequency: 1.0  # default: 0.0
planner_patience: 5.0   # default: 5.0

conservative_reset_dist: 3.0    # default: 3.0
recovery_behavior_enabled: true # default: true
clearing_rotation_allowed: true # default: true
shutdown_costmaps: false        # default: false

oscillation_timeout: 5.0    # default: 0.0
oscillation_distance: 0.2   # default: 0.5

max_planning_retries: 10    # default: -1
\end{lstlisting}

\textsf{move\_base.yml}では自律走行エージェントのマクロ的な挙動に関する数値を設定します.
各パラメータの詳細な意味については後の章で説明します.

\end{document}