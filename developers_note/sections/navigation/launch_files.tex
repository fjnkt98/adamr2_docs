\documentclass[{../../master}]{subfiles}
\graphicspath{{../../}}  % 個別コンパイル時の画像パスを解決する

\begin{document}

\section{launchファイルの記述}

Navigation Stackのノードを実行するlaunchファイルを記述します.
1つのlaunchファイルで\textsf{move\_base},\textsf{amcl},\textsf{map\_server},\textsf{rviz}を一斉に起動するようにします.

\textsf{adamr2\_navigation}パッケージの\textsf{launch/}ディレクトリに\textsf{adamr2\_navigation.launch}という名前のファイルを作成し,コード\ref{code:adamr2_navigation_launch}のように記述します.

\begin{lstlisting}[language=XML, label=code:adamr2_navigation_launch, caption=\textsf{adamr2\_navigation.launch}]
<launch>
  <!--
    Path to the map file(.yaml) for map_server node.
    When execute this launch file, it's useful to move the directory
    where the map.yaml file is located and enter as follows:
      map_file:=`pwd`/map.yaml
  -->
  <arg name="map_file" default=""/>

  <!-- Arguments for each nodes -->
  <arg name="frame_id" default="map"/>

  <arg name="amcl_param_file" default="$(find adamr2_navigation)/config/amcl/amcl.yml"/>

  <arg name="global_costmap_param_file" default="$(find adamr2_navigation)/config/costmap/global_costmap.yml"/>
  <arg name="local_costmap_param_file"  default="$(find adamr2_navigation)/config/costmap/local_costmap.yml"/>
  <arg name="global_planner_param_file" default="$(find adamr2_navigation)/config/planner/global/global_planner.yml"/>
  <arg name="local_planner_param_file"  default="$(find adamr2_navigation)/config/planner/local/trajectory_planner.yml"/>
  <arg name="move_base_param_file"      default="$(find adamr2_navigation)/config/move_base.yml"/>

  <group unless="$(eval map_file=='')">
    <!-- MAP SERVER -->
    <node name="map_server" pkg="map_server" type="map_server" args="$(arg map_file)">
      <param name="frame_id" value="$(arg frame_id)"/>
    </node>

    <!-- AMCL -->
    <node pkg="amcl" type="amcl" name="amcl" output="screen">
      <rosparam file="$(arg amcl_param_file)" command="load" />
      <remap from="scan" to="/adamr2/scan"/>
    </node>

    <!-- MOVE BASE -->
    <node pkg="move_base" type="move_base" respawn="false" name="move_base" output="screen">
      <rosparam file="$(arg global_costmap_param_file)" command="load"/>
      <rosparam file="$(arg local_costmap_param_file)"  command="load"/>
      <rosparam file="$(arg global_planner_param_file)" command="load"/>
      <rosparam file="$(arg local_planner_param_file)"  command="load"/>
      <rosparam file="$(arg move_base_param_file)"      command="load"/>

      <remap from="cmd_vel" to="/adamr2/diff_drive_controller/cmd_vel"/>
      <remap from="odom"    to="/adamr2/diff_drive_controller/odom"/>
    </node>

    <!-- Rviz -->
    <node name="rviz" pkg="rviz" type="rviz"/>
  </group>
</launch>
\end{lstlisting}

このlaunchファイルはコマンド実行時に1つの引数\textsf{map\_file}を必要とします.
Navigation Stackの実行には事前地図が必須になります.
\textsf{map\_server}ノードに地図を渡すために,SLAMで作成した地図のフルパスを引数として渡さなければなりません.
とはいえ,いちいちフルパスを入力するのは面倒なので,ソースコードに記載しているように,地図のファイル(\textsf{map.yaml}のことです.同じディレクトリに\textsf{map.pgm}も存在する必要があります)があるディレクトリをターミナルで開き,\textsf{pwd}コマンドを使ってパスを入力するやり方が便利です.
引数\textsf{map\_file}が入力されていない場合,各ノードは起動しません.
この挙動は21行目の\textsf{group}タグによるものです.

\end{document}