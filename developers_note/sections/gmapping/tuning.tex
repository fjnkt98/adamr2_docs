\documentclass[{../../master}]{subfiles}
\graphicspath{{../../}}  % 個別コンパイル時の画像パスを解決する

\begin{document}

\section{パラメータチューニング}
\label{sec:gmapping_tuning}

本節では\textsf{slam\_gmapping}のパラメータチューニングの手順の例を紹介します.
あくまでチューニングの仕方の一例であり,これが絶対的に正しいという保証をするわけではないことに注意してください.

\subsection{真っ先に変更すべきパラメータ}

\textsf{base\_frame},\textsf{map\_frame},\textsf{odom\_frame}はロボットの仕様に合わせて適切に変更してください.
\textsf{base\_frame}はADAMMR2では\textsf{''base\_footprint''}に設定しましょう.
\textsf{odom\_frame}は,\textsf{diff\_drive\_controller}が提供するTFのフレーム名が\textsf{''odom''}なので,デフォルト値のままで大丈夫です.
\textsf{map\_frame}は特別な理由がない限りはデフォルト値(\textsf{''map''})のままでいいでしょう.

\textsf{maxRange}および\textsf{maxUrange}は,使用するレーザースキャナの仕様に合わせて調整します.
RPLiDAR A2M6の公称最大測定値は\SI{18}{m}なので,\textsf{maxRange}は18.0に,\textsf{maxUrange}も18.0にします.
ただし,遠くの距離を測定したときのノイズが酷いとき等は\textsf{maxUrange}の値に少し余裕をもって,16.0とか15.0にした方がいいかもしれません.

地図の分解能を設定するパラメータである\textsf{delta}は,レーザースキャナの角度分解能が大きく影響します.
例え地図の分解能を細かく設定しても,レーザースキャナの分解能が十分に無ければ詳細な地図を作成することはできません.
RPLiDAR A2の角度分解能は標準で\SI{0.9}{deg}とされています.
この程度の分解能ならば,デフォルト値である\SI{0.05}{m}に設定しておくのが無難かと思われます.

\subsection{地図の精度を上げたいとき}

実験を通して作成した地図に歪みがあったり,地図作成に失敗するような場合は,レーザースキャナやオドメトリに関するパラメータを調整してみましょう.

\textsf{sigma}及び\textsf{lsigma}は,レーザースキャンの尤度場モデルの形状を決定するパラメータです.
\textsf{sigma}は偶然誤差のばらつきを,\textsf{lsigma}は説明できないランダム誤差の発生確率を表します.
ロボットが静止した状態で固定された壁等を計測しているときに,壁に当たっているレーザービームの測定値がばらつくときは,\textsf{sigma}の値を大きくしてみましょう.
また,障害物とロボットとの間に何もないのにその間で測定値が得られている,という現象が多く発生している場合は,\textsf{lsigma}の値を大きくしてみましょう.

また,ホイールオドメトリの精度に問題がある場合は,\textsf{srr},\textsf{srt},\textsf{str},\textsf{stt}の4つのパラメータを調整します.
これらのパラメータを0に設定すると,\textsf{slam\_gmapping}はオドメトリの情報を100\%信用するようになります.
オドメトリ誤差が大きい場合は,これらのパラメータに大きい値を設定すると\textsf{slam\_gmapping}がオドメトリ誤差を十分に考慮してくれるようになります.
大体の場合,オドメトリの誤差は回転成分が大きくずれると考えられます.
従って,回転成分の誤差のばらつきの係数である\textsf{str}および\textsf{stt}を大きめに設定すると良いでしょう.

\textsf{particles}パラメータを変更すると,使用するパーティクルの数を変更することができます.
GMappingではループとじ込みを明示的に行っていません.
しかし,スキャンマッチングによって再訪点検知を行い,再訪点に位置しているパーティクルの尤度が大きくなるように計算されています.
そのため,パーティクルの数を増やすことでループとじ込みが行われやすくなるようにすることができます.
ロボットの探索範囲が大きく,ループが多く発生するような場所の地図作成を行うときは,パーティクルの数を増やすとよいかもしれません.
ただし,パーティクルの数が増えると計算負荷もその分増えるので注意してください.

\subsection{変更の影響が不明なパラメータ}

\textsf{ogain}は尤度を平滑化するゲインであると\textsf{slam\_gmapping}のソースコード中で説明されていますが,どこで使われているのかが不明です.

\end{document}