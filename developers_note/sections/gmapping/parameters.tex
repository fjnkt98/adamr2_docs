\documentclass[{../../master}]{subfiles}
\graphicspath{{../../}}  % 個別コンパイル時の画像パスを解決する

\begin{document}

本章では\textsf{slam\_gmapping}のパラメータチューニングについて説明します.

\textsf{slam\_gmapping}を使って正確な地図を作成するには,ロボットの仕様やセンサの特性,ロボットの動作環境等に合わせた適切なパラメータを設定する必要があります.
しかし,パラメータチューニングを行うためには,GMappingがどのようなアルゴリズムで処理を行っているかをある程度把握していなければなりません.
本章では\textsf{slam\_gmapping}のパラメータチューニングを行うための情報について,GMappingのアルゴリズムやSLAMの概念を交えて解説します.
\ref{sec:gmapping_parameters}では,\textsf{slam\_gmapping}ノードが使用しているパラメータの簡単な説明を行います.
\ref{sec:gmapping_algorithm}でGMappingのアルゴリズムについて簡単な解説を行います.
\ref{sec:gmapping_tuning}では,\ref{sec:gmapping_parameters},\ref{sec:gmapping_algorithm}を踏まえて,\textsf{slam\_gmapping}のパラメータをどのように調整すればよいかを説明します.
\ref{sec:data_representation}では補助的な情報として,ROSにおけるデータの表現形式を説明します.

尚,本章で取り扱う\textsf{slam\_gmapping}のバージョンは1.4.2,本体となるGMappingのバージョンは0.2.1です.
また,本章の解説は筆者が理解できた範囲での説明となります.
間違いや説明抜け等が存在するため,あくまで参考程度にするようにしてください.

\section{\textsf{slam\_gmapping}のパラメータ}
\label{sec:gmapping_parameters}

ROS Wikiより,\textsf{slam\_gmapping}ノードが使用するパラメータを列挙します.
各パラメータは使用目的によっていくつかのグループに分類することができます.

\subsection{ROSに関するパラメータ}

\begin{itemize}
  \item \textsf{base\_frame} \\
    string型,default: \textsf{''base\_link''} \\
    ロボットのベースリンクの名前を指定します.ADAMR2では\textsf{base\_footprint}となります.
  \item \textsf{map\_frame}\\
    string型,default: \textsf{''map''} \\
    地図のフレーム名を指定します.特別なことが無い限りはデフォルト値のままで良いでしょう.
  \item \textsf{odom\_frame}\\
    string型,default: \textsf{''odom''} \\
    オドメトリ座標系のフレーム名を指定します.\textsf{diff\_drive\_controller}が提供するオドメトリ座標系は\textsf{odom}という名前なので,これを指定します.
  \item \textsf{map\_update\_interval}\\
    double型,default: 5.0 \\
    \textsf{map}トピックを配信する周期を秒単位で指定します.
    デフォルトでは5秒おきに\textsf{map}トピックが配信されます.
  \item \textsf{transform\_publish\_period}\\
    double型,default: 0.05 \\
    \textsf{slam\_gmapping}が提供する\textsf{map}→\textsf{odom}のTF変換の配信周期を指定します.
    デフォルトでは0.05秒おき,つまり\SI{20}{Hz}でTFが配信されます.
\end{itemize}

\subsection{レーザースキャンに関するパラメータ}

\begin{itemize}
  \item \textsf{maxRange} \\
    double型,default: None\\
    ロボットが使用するレーザースキャナの最大測定距離です.
    この範囲を超えたデータはすべて破棄されます.
    このパラメータが設定されていない場合は,\textsf{/scan}トピックに含まれるデータを代わりに使用します.
  \item \textsf{maxUrange} \\
    double型,default: 80.0 \\
    地図作成のために使われるスキャンデータの最大値です.
    レーザースキャンの範囲内において,障害物のない領域が地図上で自由空間として表示されるようにするには,
    \textsf{maxUrange} $<$ センサの最大測定範囲 $\leqq$ \textsf{maxRange}と設定する必要があります.
  \item \textsf{throttle\_scans} \\
    int型,default: 1 \\
    n回おきにスキャンを処理します.デフォルト値は1なので,毎回スキャンを処理します.
  \item \textsf{sigma} \\
    double型,default: 0.05 \\
    スキャンマッチングのスコアの計算に使われる分散です.
  \item \textsf{lsigma} \\
    double型,default: 0.075 \\
    レーザーセンサの確率モデルにおける,測定ノイズの分散です.
  \item \textsf{kernelSize} \\
    double型,default: 1.0 \\
    レーザーセンサの確率モデルにおける,レーザービームの終点との距離が最短となる障害物セルを探す際の探索範囲です.
\end{itemize}

GMappingではレーザーセンサの確率モデルとして,尤度場モデルが使用されています.
尤度場モデルはProbabilistic Robotics\cite{thrun2005probabilistic}にて詳しく解説されています.

\subsection{スキャンマッチングに関するパラメータ}

\begin{itemize}
  \item \textsf{lstep} \\
    double型,default: 0.05 \\
    スキャンマッチングにおける,並進方向の初期探索ステップです.
  \item \textsf{astep} \\
    double型,default: 0.05 \\
    スキャンマッチングにおける,回転方向の初期探索ステップです.
  \item \textsf{iterations} \\
    int型,default:5 \\
    スキャンマッチング処理の繰り返し回数です.
  \item \textsf{ogain} \\
    double型,default: 3.0 \\
    パーティクルの正規化の際,重みを平滑化してリサンプリングによる影響を平滑化するためのゲインです.
  \item \textsf{lskip} \\
    int型,default: 0 \\
    各スキャンにおいてスキップするレーザーの数です.デフォルト値は0なので,すべてのレーザーが計算に使用されます.
  \item \textsf{minimumScore} \\
    double型,default: 0.0 \\
    スキャンマッチングの最低スコアです.マッチングスコアがこの値以下になったとき,マッチングに失敗したとみなします.
  \item \textsf{llsamplerange} \\
    double型,default: 0.01 \\
    尤度計算の際に使用される,並進方向のサンプリング範囲です.
  \item \textsf{llsamplestep} \\
    double型,default: 0.01 \\
    尤度計算の際に使用される,並進方向のサンプリング単位です.
  \item \textsf{lasamplerange} \\
    double型,default: 0.005 \\
    尤度計算の際に使用される,回転方向のサンプリング範囲です.
  \item \textsf{lasamplestep} \\
    double型,default: 0.005 \\
    尤度計算の際に使用される,回転方向のサンプリング単位です.
\end{itemize}

\textsf{llsamplerange},\textsf{llsamplestep},\textsf{lasamplerange},\textsf{lasamplestep}の4つのパラメータのデフォルト値は,地図の分解能が\SI{5}{cm}のときの設定値のようです.
また,ソースコードを見る限り,\textsf{slam\_gmapping}においてこれらの値は使用されていません.

\subsection{状態遷移モデルに関するパラメータ}

\begin{itemize}
  \item \textsf{srr} \\
    double型,default: 0.1 \\
    ロボットの並進によって生じる,並進方向の誤差の標準偏差です.
  \item \textsf{srt} \\
    double型,default: 0.2 \\
    ロボットの回転によって生じる,並進方向の誤差の標準偏差です.
  \item \textsf{str} \\
    double型,default: 0.1 \\
    ロボットの並進によって生じる,回転方向の誤差の標準偏差です.
    \item \textsf{stt} \\
    double型,default: 0.2 \\
    ロボットの回転によって生じる,回転方向の誤差の標準偏差です.
\end{itemize}

これら4つのパラメータは,オドメトリに基づいた状態遷移モデルにおいて使用される数値です.
オドメトリに基づいた状態遷移モデルについては,Probabilistic Robotics\cite{thrun2005probabilistic}に詳細が記載されています.

\subsection{パーティクルフィルタに関するパラメータ}

\begin{itemize}
  \item \textsf{linearUpdate} \\
    double型,default: 1.0 \\
    パーティクルフィルタの更新が行われるために必要なロボットの並進方向の移動距離です.
    ロボットがこれだけ移動しないと,フィルタのアップデートは行われません.
  \item \textsf{angularUpdate} \\
    double型,default: 0.5 \\
    パーティクルフィルタの更新が行われるために必要なロボットの回転方向の移動距離です.
    ロボットがこれだけ移動しないと,フィルタのアップデートは行われません.
  \item \textsf{temporalUpdate} \\
    double型,default: -1.0 \\
    時間経過によるフィルタアップデートを許可し,その周期を指定します.
    ロボットが一定距離を移動しなくても,このパラメータで指定した時間が経過したらフィルタのアップデートが行われます.
    マイナスの値が設定されていれば,時間経過によるフィルタアップデートは行われなくなります.
  \item \textsf{resampleThreshold} \\
    double型,default: 0.5 \\
    リサンプリングを行うかどうかを決定するための閾値です.
  \item \textsf{particles} \\
    int型,default: 30 \\
    パーティクルフィルタで使用するパーティクルの数です.
    GMappingではパーティクルの数は固定値です.
\end{itemize}

\subsection{地図に関するパラメータ}

\begin{itemize}
  \item \textsf{xmin} \\
    double型,default: -100.0 \\
    地図のX方向の最小値の初期値です.
  \item \textsf{ymin} \\
    double型,default: -100.0 \\
    地図のY方向の最小値の初期値です.
  \item \textsf{xmax} \\
    double型,default: 100.0 \\
    地図のX方向の最大値の初期値です.
  \item \textsf{ymax} \\
    double型,default: 100.0 \\
    地図のY方向の最大値の初期値です.
  \item \textsf{delta} \\
    double型,default: 0.05 \\
    地図の分解能です.占有格子地図のグリッド1つあたりの長さをメートル単位で指定します.
    デフォルト値は0.05なので,グリッド1つあたりの長さは\SI{5}{cm}となります.
  \item \textsf{occ\_thresh} \\
    double型,default: 0.25 \\
    占有格子地図における,障害物の閾値です.
    この値以上の値を持つセルを障害物とみなします.
\end{itemize}

地図の範囲を表すパラメータに小さい値を指定していたとしても,地図のサイズはロボットの移動とフィルタのアップデートに伴って更新されます.

\end{document}