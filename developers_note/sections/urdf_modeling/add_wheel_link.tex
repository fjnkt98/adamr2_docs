\documentclass[{../../master}]{subfiles}
\graphicspath{{../..}}  % 個別コンパイル時の画像パスを解決する

\begin{document}

\section{\textsf{wheel\_link}の作成と追加}

最後に\textsf{wheel\_link}とそのジョイントを追加していきます.
\textsf{wheel\_link}を定義するために2つのファイルを準備します.
\textsf{wheel.xacro}と\textsf{transmission.xacro}の2つです.

\subsection{\textsf{wheel.xacro}の作成}

まずは\textsf{wheel.xacro}を記述していきます.
\textsf{urdf/}ディレクトリ以下に\textsf{wheel/}ディレクトリを作成し,\textsf{wheel.xacro}という名前のファイルを作成します.
そして,コード\ref{code:wheel_xacro}のように記述します.

\begin{lstlisting}[language=XML, label=code:wheel_xacro, caption=\textsf{wheel.xacro}]
<?xml version="1.0"?>
<robot xmlns:xacro="http://ros.org/wiki/xacro">
  <xacro:macro name="wheel" params="prefix parent *joint_origin *joint_axis">
    <joint name="${prefix}_wheel_joint" type="continuous">
      <xacro:insert_block name="joint_origin"/>
      <parent link="${parent}"/>
      <child link="${prefix}_wheel_link"/>
      <xacro:insert_block name="joint_axis"/>
    </joint>
    
    <link name="${prefix}_wheel_link">
      <visual>
        <origin rpy="0.0 0.0 0.0"/>
        <geometry>
          <mesh filename="package://adamr2_description/meshes/wheel_link.STL"/>
        </geometry>
        <material name="black">
          <color rgba="0.0 0.0 0.0 1.0"/>
        </material>
      </visual>
    </link>
  </xacro:macro>
</robot>
\end{lstlisting}

\textsf{wheel.xacro}の内容は他のリンクとほとんど変わり映えしません.
ジョイントとリンクを定義しているだけです.
ホイールを繋ぐジョイントはタイプを\textsf{continuous}にします.

\subsection{\textsf{transmission.xacro}の作成}

ホイールジョイントに対する\textsf{ros\_control}の設定を行うマクロは\textsf{transmision.xacro}で定義します.
\textsf{wheel.xacro}と同じディレクトリに\textsf{transmission.xacro}という名前のファイルを作成し,コード\ref{code:transmission_xacro}のような内容を記述します.

\begin{lstlisting}[language=XML, label=code:transmission_xacro, caption=\textsf{transmission.xacro}]
<?xml version="1.0"?>
<robot xmlns:xacro="http://ros.org/wiki/xacro">
  <xacro:macro name="wheel_trans" params="prefix">
    <transmission name="${prefix}_wheel_trans">
      <type>transmission_interface/SimpleTransmission</type>
      <joint name="${prefix}_wheel_joint">
        <hardwareInterface>hardware_interface/VelocityJointInterface</hardwareInterface>
      </joint>

      <actuator name="${prefix}_wheel_motor">
        <mechanicalReduction>1</mechanicalReduction>
      </actuator>
    </transmission>
  </xacro:macro>
</robot>
\end{lstlisting}

このマクロではホイールのジョイントに対して\textsf{transmission}要素を追加しています.
\textsf{transmission}要素には\textsf{type},\textsf{joint},\textsf{actuator}の3つの要素を持っています.
このURDFは\textsf{ros\_control}のコントローラの1つである\textsf{diff\_drive\_controller}から利用されることを想定しているので,
\textsf{type}は\textsf{transmission\_interface/SimpleTransmission}を,
\textsf{joint}の\textsf{hardwareInterface}要素には\textsf{hardware\_interface/VelocityJointInterface}を設定しています.
\textsf{actuator}要素の\textsf{mechanicalReduction}には,ロボットのアクチュエータの減速比を設定します.
ADAMR2ではギヤードBLDCモータを使用していますが,ギア比の計算はYP-Spurが行ってくれるため,ここで設定する必要はありません.
ギヤ比は1に設定しておきましょう.

\end{document}