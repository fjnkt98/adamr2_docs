\documentclass[{../../master}]{subfiles}
\graphicspath{{../..}}  % 個別コンパイル時の画像パスを解決する

\begin{document}

\section{\textsf{base\_link}の作成と追加}

節\ref{sec:create_urdf_package}までの作業で,ルートファイル\textsf{robot.xacro}の作成と\textsf{base\_footprint}リンクの定義を行いました.
この節ではロボットのボディのリンクである\textsf{base\_link}の定義とルートファイルへの追加の作業を行います.

\subsection{\textsf{base.xacro}の作成}

まず,\textsf{base\_link}の要素や属性の定義を記述するファイルを作成します.
\textsf{urdf/}ディレクトリ以下に\textsf{base/}ディレクトリを作成し,\textsf{base.xacro}という名前のファイルを作成します.
そして,まずはコード\ref{code:base_link_first_step}のような内容を記述します.

\begin{lstlisting}[language=XML, caption=\textsf{base.xacro}, label=code:base_link_first_step]
<?xml version="1.0"?>
<robot xmlns:xacro="http://ros.org/wiki/xacro">
  <xacro:macro name="base" params="parent *joint_origin">
    
  </xacro:macro>
</robot>
\end{lstlisting}

ルートファイルと同じく,\textsf{robot}タグをトップレベルに記述し,その中に他の全ての要素を記述していきます.
ただし,コンポーネントファイルには\textsf{name}属性を書く必要はありません.名前空間だけ指定しておきましょう.

\textsf{robot}タグの中には,\textsf{xacro:macro}タグが書かれています.
これが\textsf{xacro}におけるマクロの定義であり,ルートファイルから呼び出されるものです.
マクロの中に\textsf{link}や\textsf{joint}等のタグを記述しておけば,ルートファイルから呼び出されたときにそれらが展開される,という仕組みです.
\textsf{xacro:macro}タグには\textsf{name}属性と\textsf{params}属性があります.
\textsf{name}属性にはマクロの名前を指定します.
\textsf{params}属性には,そのマクロが取る引数を指定することができます.
コード\ref{code:base_link_first_step}では,引数として\textsf{parent}(親リンク名の指定)と\textsf{*joint\_origin}
\footnote{アスタリスクが先頭に付くパラメータは「ブロックパラメータ」と呼ばれるパラメータで,任意の要素を展開することができます.}
(座標原点の指定)の2つを取るように設定しています.

\subsection{\textsf{link}タグの設定}



\subsection{メッシュファイル(STL形式)の作成と割り当て}

\subsection{\textsf{joint}の設定}

\subsection{ファイルのインクルード}

\subsection{\textsf{rviz}による可視化}

\end{document}