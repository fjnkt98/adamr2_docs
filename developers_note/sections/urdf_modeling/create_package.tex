\documentclass[{../../master}]{subfiles}
\graphicspath{{../..}}  % 個別コンパイル時の画像パスを解決する

\begin{document}

\section{ディレクトリ構成}

慣習的に,ロボットのURDFモデルは\textsf{\*\_description}(\* はロボットもしくはプロジェクトの名前)という名前のROSパッケージに配置します.
ADAMR2プロジェクトならば「\textsf{adamr2\_description}」という名前になります.
URDFモデリングを始める前に,パッケージを作っておきましょう.
ビルド依存パッケージは無いので,パッケージ作成コマンドにオプションは必要ありません.

\begin{lstlisting}[language=sh, caption=Create a Package to put the URDF Model in]
catkin create pkg adamr2_description
\end{lstlisting}

\textsf{adamr2\_description}パッケージのディレクトリ構成はコード\ref{code:package_directory_structure}のようになります.

\begin{lstlisting}[language=sh, caption=Directory Structure of adamr2\_description, label=code:package_directory_structure]
adamr2_description/
  ├ launch/
  ├ meshes/
  ├ rviz/
  ├ urdf/
  ├ package.xml
  └ CMakeLists.txt
\end{lstlisting}

\textsf{launch/}ディレクトリには,URDFモデルの可視化を行うためのlaunchファイルを置きます.
\textsf{meshes/}ディレクトリには,STL形式のロボットの3Dモデルを置きます.
\textsf{rviz/}ディレクトリには,URDFモデル可視化時の\textsf{rviz}の設定ファイルを置きます.
そして,\textsf{urdf/}ディレクトリにxacroファイルを格納していきます.

\textsf{urdf/}ディレクトリの中身のディレクトリ構造はコードのようになります.

\begin{lstlisting}[language=sh, caption=Directory Structure of \textsf{urdf/}, label=code:urdf_directory_structure]
urdf/
  ├ base/
  │   └ base.xacro
  ├ caster/
  │   └ caster.xacro
  ├ lidar/
  │   └ lidar.xacro
  ├ wheel/
  │   ├ transmission.xacro
  │   └ wheel.xacro
  └ robot.xacro
\end{lstlisting}

\textsf{robot.xacro}がルートファイルです.
ルートファイルから各コンポーネントのxacroファイルをインクルードし,ロボットのモデルを作成します.
このファイルは最終的に\textsf{xacro}パッケージのノードを用いてURDFファイルに変換されます.

ロボットの主要コンポーネントである\textsf{base},\textsf{caster},\textsf{lidar},\textsf{wheel}のそれぞれに対してディレクトリを作り,その中に各々のxacroファイルを格納します.
\textsf{caster}及び\textsf{wheel}はロボットに2つずつ存在しますが,xacroファイルを複数作成する必要はありません.
共通のモジュールとしてxacroファイルを作成しておき,ルートファイルから読み込む際に名前や位置を付けることで各リンクを定義します.
また,\textsf{wheel}には\textsf{transmission.xacro}というファイルがありますが,これは\textsf{ros\_control}を用いたコントローラを作る際に必要となるオプションを設定するためのファイルです.


\end{document}