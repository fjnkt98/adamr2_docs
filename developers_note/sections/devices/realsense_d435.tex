\documentclass[{../../master}]{subfiles}
\graphicspath{{../../}}  % 個別コンパイル時の画像パスを解決する

\begin{document}

\section{Realsense D435を利用する}

\subsection{Realsense D435について}

Intel Realsense Depth Camera D435はIntel社が開発している深度カメラです.
Realsense D435を使うことで,画像データと3次元点群データを取得することができます.
通常のカメラとして使うことができる他,点群データを用いてVisual SLAMを行ったり,障害物検知を行ったりすることも可能です.

Realsense D435のROSラッパーはIntel社によって用意されているので,簡単に使用することができます.
本節ではRealsense D435のROSでの使い方を説明します.

\subsection{ライブラリのインストール}

ROSラッパーは\textsf{apt}で公開されているので,簡単にインストールすることができます
Ubuntu 18.04LTS(ROS Melodic)の場合,以下のコマンドでインストールします.

\begin{lstlisting}[language=sh, caption=Install Realsense ROS Wrapper]
sudo apt update
sudo apt install ros-melodic-realsense2-camera ros-melodic-realsense2-description
\end{lstlisting}

ROSパッケージをインストールすることで,RealsenseのドライバやIntel公式ビューワー\textsf{realsense-viewer}もインストールされます.
データの確認やファームウェアの更新は\textsf{realsense-viewer}から行えます.
ターミナルで\textsf{realsense-viewer}と入力して実行することでビューワーを起動することができます.
PCにRealsenseを接続し,ビューワーからカメラをオンにすると点群データを確認できます.

\subsection{Realsense起動用launchファイルを準備する}

Realsenseを起動するためのlaunchファイルを記述します.
\textsf{adamr2\_bringup}パッケージの\textsf{launch/}ディレクトリに\textsf{realsense.launch}という名前のファイルを作成し,コード\ref{code:realsense_launch}のように記述します.

\begin{lstlisting}[language=XML, label=code:realsense_launch, caption=\textsf{realsense.launch}]
<launch>
  <arg name="serial_no"           default=""/>
  <arg name="usb_port_id"         default=""/>
  <arg name="device_type"         default=""/>
  <arg name="json_file_path"      default=""/>
  <arg name="camera"              default="camera"/>
  <arg name="tf_prefix"           default="$(arg camera)"/>
  <arg name="external_manager"    default="false"/>
  <arg name="manager"             default="realsense2_camera_manager"/>

  <arg name="fisheye_width"       default="640"/>
  <arg name="fisheye_height"      default="480"/>
  <arg name="enable_fisheye"      default="false"/>

  <arg name="depth_width"         default="640"/>
  <arg name="depth_height"        default="480"/>
  <arg name="enable_depth"        default="true"/>

  <arg name="infra_width"        default="640"/>
  <arg name="infra_height"       default="480"/>
  <arg name="enable_infra"        default="false"/>
  <arg name="enable_infra1"       default="false"/>
  <arg name="enable_infra2"       default="false"/>

  <arg name="color_width"         default="640"/>
  <arg name="color_height"        default="480"/>
  <arg name="enable_color"        default="true"/>

  <arg name="fisheye_fps"         default="30"/>
  <arg name="depth_fps"           default="30"/>
  <arg name="infra_fps"           default="30"/>
  <arg name="color_fps"           default="30"/>
  <arg name="gyro_fps"            default="400"/>
  <arg name="accel_fps"           default="250"/>
  <arg name="enable_gyro"         default="false"/>
  <arg name="enable_accel"        default="false"/>

  <arg name="enable_pointcloud"         default="true"/>
  <arg name="pointcloud_texture_stream" default="RS2_STREAM_COLOR"/>
  <arg name="pointcloud_texture_index"  default="0"/>

  <arg name="enable_sync"               default="true"/>
  <arg name="align_depth"               default="false"/>

  <arg name="publish_tf"                default="true"/>
  <arg name="tf_publish_rate"           default="0"/>

  <arg name="filters"                   default="pointcloud"/>
  <arg name="clip_distance"             default="-2"/>
  <arg name="linear_accel_cov"          default="0.01"/>
  <arg name="initial_reset"             default="false"/>
  <arg name="unite_imu_method"          default=""/>
  <arg name="topic_odom_in"             default="odom_in"/>
  <arg name="calib_odom_file"           default=""/>
  <arg name="publish_odom_tf"           default="true"/>
  <arg name="allow_no_texture_points"   default="false"/>

  <group ns="/adamr2/$(arg camera)">
    <include file="$(find realsense2_camera)/launch/includes/nodelet.launch.xml">
      <arg name="tf_prefix"                value="$(arg tf_prefix)"/>
      <arg name="external_manager"         value="$(arg external_manager)"/>
      <arg name="manager"                  value="$(arg manager)"/>
      <arg name="serial_no"                value="$(arg serial_no)"/>
      <arg name="usb_port_id"              value="$(arg usb_port_id)"/>
      <arg name="device_type"              value="$(arg device_type)"/>
      <arg name="json_file_path"           value="$(arg json_file_path)"/>

      <arg name="enable_pointcloud"        value="$(arg enable_pointcloud)"/>
      <arg name="pointcloud_texture_stream" value="$(arg pointcloud_texture_stream)"/>
      <arg name="pointcloud_texture_index"  value="$(arg pointcloud_texture_index)"/>
      <arg name="enable_sync"              value="$(arg enable_sync)"/>
      <arg name="align_depth"              value="$(arg align_depth)"/>

      <arg name="fisheye_width"            value="$(arg fisheye_width)"/>
      <arg name="fisheye_height"           value="$(arg fisheye_height)"/>
      <arg name="enable_fisheye"           value="$(arg enable_fisheye)"/>

      <arg name="depth_width"              value="$(arg depth_width)"/>
      <arg name="depth_height"             value="$(arg depth_height)"/>
      <arg name="enable_depth"             value="$(arg enable_depth)"/>

      <arg name="color_width"              value="$(arg color_width)"/>
      <arg name="color_height"             value="$(arg color_height)"/>
      <arg name="enable_color"             value="$(arg enable_color)"/>

      <arg name="infra_width"              value="$(arg infra_width)"/>
      <arg name="infra_height"             value="$(arg infra_height)"/>
      <arg name="enable_infra"            value="$(arg enable_infra)"/>
      <arg name="enable_infra1"            value="$(arg enable_infra1)"/>
      <arg name="enable_infra2"            value="$(arg enable_infra2)"/>

      <arg name="fisheye_fps"              value="$(arg fisheye_fps)"/>
      <arg name="depth_fps"                value="$(arg depth_fps)"/>
      <arg name="infra_fps"                value="$(arg infra_fps)"/>
      <arg name="color_fps"                value="$(arg color_fps)"/>
      <arg name="gyro_fps"                 value="$(arg gyro_fps)"/>
      <arg name="accel_fps"                value="$(arg accel_fps)"/>
      <arg name="enable_gyro"              value="$(arg enable_gyro)"/>
      <arg name="enable_accel"             value="$(arg enable_accel)"/>

      <arg name="publish_tf"               value="$(arg publish_tf)"/>
      <arg name="tf_publish_rate"          value="$(arg tf_publish_rate)"/>

      <arg name="filters"                  value="$(arg filters)"/>
      <arg name="clip_distance"            value="$(arg clip_distance)"/>
      <arg name="linear_accel_cov"         value="$(arg linear_accel_cov)"/>
      <arg name="initial_reset"            value="$(arg initial_reset)"/>
      <arg name="unite_imu_method"         value="$(arg unite_imu_method)"/>
      <arg name="topic_odom_in"            value="$(arg topic_odom_in)"/>
      <arg name="calib_odom_file"          value="$(arg calib_odom_file)"/>
      <arg name="publish_odom_tf"          value="$(arg publish_odom_tf)"/>
      <arg name="allow_no_texture_points"  value="$(arg allow_no_texture_points)"/>
    </include>
  </group>
</launch>
\end{lstlisting}

\textsf{realsense2\_camera}パッケージの\textsf{nodelet.launch.xml}ファイルをインクルードし,パラメータを渡して起動します.
\footnote{パラメータが直書きなのはYAMLファイルを介してパラメータを渡す方法がイマイチ良くわからなかったからです.}
このlaunchファイルはサンプルとほとんど変更点はありませんが,ノードを\textsf{adamr2}名前空間に置いている点が異なります.
これにより,Realsenseが配信するトピックが\textsf{/adamr2/**}のように\textsf{adamr2}名前空間の下に置かれるようになります.

\subsection{Realsenseを起動してみる}

コード\ref{code:realsense_launch}のlaunchファイルを実行することでRealsense D435をROS上で利用することができます.
Realsense D435をPCに接続し,コード\ref{code:execute_realsense_launch}のようにコマンドを実行します.
RPLIDARのようにデバイスファイルを指定する必要はありません.

\begin{lstlisting}[language=sh, label=code:execute_realsense_launch, caption=Launch \textsf{realsense.launch}]
roslaunch adamr2_bringup realsense.launch
\end{lstlisting}

Realsenseの起動に成功すると,トピックが配信されます.
\textsf{rviz}から確認してみる等してみましょう.

Realsenseはたまに起動に失敗することがあります.
Reanselseは省電力とはいえ結構電流を食うので,バスパワーUSBハブ等には繋がず,PCのUSBポートに直接接続する方が望ましいです.
また,ファームウェアのバージョンが古いと問題が発生することがあります.
その場合は\textsf{realsense-viewer}からファームウェアの更新を行ってください.

\end{document}