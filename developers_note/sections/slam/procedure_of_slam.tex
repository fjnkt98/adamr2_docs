\documentclass[{../../master}]{subfiles}
\graphicspath{{../../}}  % 個別コンパイル時の画像パスを解決する

\begin{document}

\section{SLAMの手順}

SLAMを実行する前に,ROSにおけるSLAMの手順をおさらいしておきます.

SLAMの実験のシチュエーションは,オンラインSLAMとオフラインSLAMの2つがあります.
両者の違いは,SLAMの処理をロボットの動作中に行うか否かにあります.
オンラインSLAMではロボットの走行と同時にSLAMの処理を実行します.
地図作成をリアルタイムに行うことになるので,処理の負荷はかなり大きいです.
一方,オフラインSLAMは,ロボットの走行データ(オドメトリやレーザースキャン)を記録・保存しておき,そのデータを再生しながらSLAMの処理を実行する,いわゆるバッチ処理を行います.
走行データを記録するのとSLAMの処理を実行するのとを分けて行うため時間と手間がかかりますが,後から処理を行えるため計算負荷が比較的小さいこと,走行データが不変なためパラメータチューニングが行いやすいこと,他のSLAMの手法と比較しやすいこと等が利点です.

ここではオフラインSLAMの手順について説明します.

\subsection{走行データの記録}

まずは走行データの記録から始めます.
ROSでは\textsf{rosbag}\footnote{\url{http://wiki.ros.org/rosbag}}という,トピックメッセージを記録するためのツールが用意されています.
これを使って走行データを記録します.

SLAMに必要なデータは,\ref{sec:required_things_for_slam}でも言及した通り,\textsf{/tf}トピックと\textsf{/scan}トピックです.
また,SLAMの処理には必要ありませんが,SLAM実行時にロボットのモデルを可視化しておくと何かと便利なので,\textsf{/tf\_static}と\textsf{/odom}トピックも記録しておくと良いでしょう.

まずはロボットを実行します.コード\ref{code:bring_up_adamr2_for_slam}を実行して,\ref{chap:controller_implementation}で実装したコントローラを起動します.

\begin{lstlisting}[language=sh, label=code:bring_up_adamr2_for_slam, caption=Bring up ADAMR2]
roslaunch adamr2_control adamr2_control.launch
\end{lstlisting}

ロボットを手動操縦するので,ジョイスティックも起動しましょう.
コード\ref{bring_up_joy_for_slam}を実行して,ジョイスティックを起動します.

\begin{lstlisting}[language=sh, label=code:bring_up_joy_for_slam, caption=Bring up Joystick]
roslaunch adamr2_bringup joy.launch
\end{lstlisting}

ロボットを起動したら,\textsf{rosbag}を使ってトピックを記録します.
\textsf{rosbag}はコマンドラインから利用することができます.
SLAMのためのトピックを記録するには,コード\ref{code:rosbag_record_for_slam}を実行します.

\begin{lstlisting}[language=sh, label=code:rosbag_record_for_slam, caption=Execute \textsf{rosbag} for SLAM]
rosbag record -O slam.bag /tf /tf_static /adamr2/scan /adamr2/diff_drive_controller/odom
\end{lstlisting}

\textsf{-O}オプションを付けることで,Bagファイル\footnote{\textsf{rosbag record}コマンドで保存したファイルのことをBagファイルと呼びます.}の名前を指定することができます.
コード\ref{code:rosbag_record_for_slam}では「\textsf{slam.bag}」という名前を指定しています.

コード\ref{code:rosbag_record_for_slam}を実行すると,トピックの記録が始まります.
実行中の\textsf{rosbag}を\textsf{Ctrl-C}で停止することで,トピックの記録を停止,データを保存することができます.
記録したデータは\textsf{rosbag record}コマンドを実行したディレクトリに,1つのファイルとして保存されます.
ファイル名はデフォルトでは日付と時刻が使われます.
ファイル名を指定したい場合は,\textsf{-O}オプションを付けて\textsf{rosbag}を実行します.
詳しくは\textsf{rosbag}のROS Wikiを参照してください.

コード\ref{code:rosbag_record_for_slam}を実行してから,地図を作成したい範囲をロボットに走行させます.
障害物や壁等,地図作成に必要となる情報をできるだけ多く取れるよう,工夫して走行させる必要があります.

十分なデータが取れたら,\textsf{rosbag}を終了してデータを保存します.
データを保存した後はロボットは動かす必要がないので,適切に終了させます.

念のためちゃんとデータが記録できたかどうかを確認しておきましょう.
Bagファイルの情報をチェックするには,\textsf{rosbag info}コマンドを使用します.
コード\ref{code:rosbag_info}を実行することでチェックを行えます.

\begin{lstlisting}[language=sh, label=code:rosbag_info, caption=Check the Bag File by \textsf{rosbag info}]
rosbag info slam.bag
\end{lstlisting}

\textsf{rosbag info}を実行すると,Bagファイルに記録されたトピック,及びそのデータ型,データの合計時間等の情報が表示されます.
データの抜けが無いか等を確認しましょう.

\subsection{SLAMの実行}

ここからはPC上での作業のみとなるので,ロボットを動かす必要はありません.

まずはROS Masterを起動します.

\begin{lstlisting}[language=sh, caption=Execute \textsf{roscore}]
roscore
\end{lstlisting}

\textsf{roscore}を実行したのとは別のターミナルを開き,\textsf{use\_sim\_time}パラメータをtrueに設定します.
これは,\textsf{rosbag}のファイルを再生して処理を行う際に必要となるオプションです.

\begin{lstlisting}[language=sh, caption=Set \textsf{use\_sim\_time} to true]
rosparam use_sim_time true
\end{lstlisting}

更にもう1つのターミナルを開き,\textsf{gmapping}を実行します.
先ほど作成したlaunchファイルを実行します.
\textsf{gmapping}はデータが送られない限り処理を行わないので,Bagファイルを再生する前に起動しておきます.

\begin{lstlisting}[language=sh, caption=Run \textsf{gmapping.launch}]
roslaunch adamr2_slam gmapping.launch
\end{lstlisting}

更にもう1つターミナルを開き,\textsf{rviz}を起動します.

\begin{lstlisting}[language=sh, caption=Run \textsf{rviz}]
rosrun rviz rviz
\end{lstlisting}

\textsf{rviz}では「map」,「LaserScan」,「RobotModel」及び「TF」を表示するように設定します.

最後に,Bagファイルを再生します.
ターミナルを新しく開き,Bagファイルがあるディレクトリに移動し,コード\ref{code:replay_bag_file_for_slam}を実行します.

\begin{lstlisting}[language=sh, label=code:replay_bag_file_for_slam, caption=Replay \textsf{slam.bag}]
rosbag play --clock slam.bag
\end{lstlisting}

Bagファイルの再生が開始され,トピックのデータが配信されると,\textsf{gmapping}が処理を開始します.
\textsf{rviz}で適切にトピックの表示設定を行うと,地図が逐次作成されることがわかります.

地図の作成が終わったら,地図を保存します.
地図の保存は\textsf{map\_server}パッケージ\footnote{\url{http://wiki.ros.org/map_server}}の\textsf{map\_saver}ノードによって行います.
名前が似ていて非常に紛らわしいので,間違えないように注意してください.
新しいターミナルを開き,コード\ref{code:save_map}を実行します.

\begin{lstlisting}[language=sh, label=code:save_map, caption=Save the Map]
rosrun map_server map_saver
\end{lstlisting}

コード\ref{code:save_map}を実行したディレクトリに,「\textsf{map.pgm}」と「\textsf{map.yaml}」の2つのファイルが生成されます.
\textsf{map.pgm}はPortable Gray Mapファイルで,地図の本体です.
\textsf{map.yaml}は地図の解像度や範囲などのメタ情報を保存するファイルです.
これらのファイルは2つで1組であり,同じディレクトリ内に存在していなければなりません.

以上でオフラインSLAMの手順は終了です.後は作成した地図をもとに,Navigation等を行えるようになります.

なお,オンラインSLAMを行いたい場合は,コード\ref{code:gmapping_launch}の\textsf{gmapping.launch}の中の\textsf{robot\_description}に関する\textsf{param}タグを削除し,
\footnote{\textsf{adamr2\_control.launch}で\textsf{robot\_description}の登録を行っているので,オンラインSLAMの場合はこの記述が余計になるからです.逆に,オフラインSLAMでは\textsf{adamr2\_control.launch}を起動しないからこの記述が必要になっている訳です.}
ロボットの起動中に\textsf{gmapping.launch}を実行すれば良いです.
作成した地図は同じく\textsf{map\_saver}によって保存することができます.

\end{document}