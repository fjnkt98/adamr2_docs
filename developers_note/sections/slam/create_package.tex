\documentclass[{../../master}]{subfiles}
\graphicspath{{../../}}  % 個別コンパイル時の画像パスを解決する

\begin{document}

\section{SLAMのためのパッケージ作成}

\subsection{パッケージの作成}

\textsf{slam\_gmapping}を実行するためのパッケージを作成します.
名前は\textsf{adamr2\_slam}とします.
コード\ref{code:create_adamr2_slam_package}を実行して,パッケージを作成します.
依存パッケージの指定はここではしなくても構いません.

\begin{lstlisting}[language=sh, label=code:create_adamr2_slam_package, caption=Create \textsf{adamr2\_slam} Package]
catkin create pkg adamr2_slam
\end{lstlisting}

このパッケージには\textsf{slam\_gmapping}のノードを実行するためのlaunchファイルと,ノードに渡すパラメータを記述したパラメータファイルを格納します.
ディレクトリ構成はコード\ref{code:directory_structure_of_adamr2_slam}のようになります.

\begin{lstlisting}[language=sh, caption=Directory Structure of adamr2\_slam, label=code:directory_structure_of_adamr2_slam]
adamr2_slam/
  ├ launch/
  ├ config/
  ├ package.xml
  └ CMakeLists.txt
\end{lstlisting}

\textsf{launch/}ディレクトリには各種launchファイルを格納します.
\textsf{config/}ディレクトリにはパラメータファイルを格納します.

パッケージが作成できたら,次はlaunchファイルとパラメータファイルの準備に入ります.

\subsection{launchファイルの作成}

\textsf{slam\_gmapping}のノードを実行するためのlaunchファイルを作成します.
\textsf{adamr2\_slam}パッケージの\textsf{launch/}ディレクトリ内に\textsf{gmapping.launch}という名前のファイルを作成し,コード\ref{code:gmapping_launch}のように記述します.

\begin{lstlisting}[language=XML, label=code:gmapping_launch, caption=\textsf{gmapping.launch}]
<launch>
  <param name="robot_description" command="$(find xacro)/xacro $(arg model)"/>

  <node pkg="gmapping" type="slam_gmapping" name="slam_gmapping" clear_params="true">
    <rosparam file="$(find adamr2_slam)/config/gmapping/gmapping.yml" command="load"/>
    <remap from="scan" to="/adamr2/scan"/>
  </node>
</launch>
\end{lstlisting}

1つのノードのみを実行する単純なlaunchファイルです.
\textsf{rosparam}タグを使って,パラメータファイルに記述したパラメータを一括でノードに渡しています.
パラメータファイルは次の小節で作成するものです.
また,\ref{sec:prepare_rplidar_launch}でRPLiDARが出力する\textsf{sensor\_msgs/LaserScan}型メッセージのトピックに名前空間を設定していたため,それを\textsf{slam\_gmapping}ノードが受け取れるように\textsf{remap}タグを使ってトピック名を変更しています.
また,ロボットのURDFの情報をROSに登録する処理も行っています.
これは,後述するオフラインSLAMにおいて,ロボットのモデルを可視化するために必要な処理です.

\subsection{パラメータファイルの作成}

次に,パラメータファイルを作成します.
\textsf{config/}ディレクトリ内に\textsf{gmapping.yml}という名前のファイルを作成し,コード\ref{code:gmapping_yml}のように記述します.

\begin{lstlisting}[language=YAML, label=code:gmapping_yml, caption=\textsf{gmapping.yml}]
# Basic Parameters
throttle_scans: 1               # default = 1
base_frame: base_footprint      # default = base_link
map_frame: map                  # default = map
odom_frame: odom                # default = odom
map_update_interval: 5.0        # default = 5.0
transform_publish_period: 0.05  # default = 0.05

# Laser Parameters
maxRange: 18
maxUrange: 15
sigma: 0.05       # default = 0.05
kernelSize: 1     # default = 1
lstep: 0.05       # default = 0.05
astep: 0.05       # default = 0.05
iterations: 5     # default = 5
lsigma: 0.075     # default = 0.075
ogain: 3.0        # default = 3.0
lskip: 0          # default = 0
minimumScore: 0.0 # default = 0.0

# Motion Model Parameters
srr: 0.1  # default = 0.1
srt: 0.2  # default = 0.2
str: 0.1  # default = 0.1
stt: 0.2  # default = 0.2

# Particle Filter Parameters
linearUpdate: 1.0       # default = 1.0
angularUpdate: 0.5      # default = 0.5
temporalUpdate: -1.0    # default = -1.0
resampleThreshold: 0.5  # default = 0.5
particles: 30           # default = 30

# Map Parameters
xmin: -100.0      # default = -100.0
ymin: -100.0      # default = -100.0
xmax: 100.0       # default = 100.0
ymax: 100.0       # default = 100.0
delta: 0.05       # default = 0.05
occ_thresh: 0.25  # default = 0.25

# Likelihood Parameters
llsamplerange: 0.01   # default = 0.01
llsamplestep: 0.01    # default = 0.01
lasamplerange: 0.005  # default = 0.005
lasamplestep: 0.005   # default = 0.005
\end{lstlisting}

ほとんどデフォルト値に設定しています.
ADAMR2ではベースリンクを\textsf{base\_footprint}にしているため,\textsf{base\_frame}のみ変更しています.
また,レーザーセンサの最大レンジを示す\textsf{maxRange}と,地図作成に使用するレーザーの最大距離を示す\textsf{maxUrange}については,使用するセンサに従って変更する必要があります.
筆者の環境ではRPLiDAR A2M6を使用しており,このモデルは最大測定距離が\SI{18}{m}のため,\textsf{maxRange}を18に,\textsf{maxUrange}を15に設定しています.

\subsection{\textsf{package.xml}の編集}

このパッケージは\textsf{slam\_gmapping}のノードを実行するため,依存パッケージに\textsf{slam\_gmapping}を追加しておきます.

\textsf{package.xml}をコード\ref{code:package_xml_of_adamr2_slam}のように編集します.

\begin{lstlisting}[language=XML, label=code:package_xml_of_adamr2_slam, caption=\textsf{package.xml}]
<?xml version="1.0"?>
<package format="2">
  <name>adamr2_slam</name>
  <version>0.0.0</version>
  <description>The adamr2_slam package</description>

  <maintainer email="hoge@hogehoge.com">hoge</maintainer>

  <license>BSD</license>

  <buildtool_depend>catkin</buildtool_depend>
  <exec_depend>slam_gmapping</exec_depend>
</package>
\end{lstlisting}

実行時に\textsf{slam\_gmapping}を使用するため,\textsf{exec\_depend}タグを使って\textsf{slam\_gmapping}パッケージを指定しています.

\end{document}