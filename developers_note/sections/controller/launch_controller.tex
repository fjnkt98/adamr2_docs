\documentclass[{../../master}]{subfiles}
\graphicspath{{../../}}  % 個別コンパイル時の画像パスを解決する

\begin{document}

\section{コントローラの起動}

ここまでの作業で,\textsf{ros\_control}を使用したロボットの開発のための要素は揃いました.
後は\textsf{adamr2\_control.launch}を実行することで,コントローラが起動するはずです.

\subsection{デバイスの接続}

コントローラを起動するためには,デバイスをPCに繋いで,動作可能な状態にしておく必要があります.
二軸ブラシレスモータドライバTF-2MD-R6に電源を接続し,PCに接続します.
デバイスがPCから認識されていることを確認するには,コード\ref{code:check_driver_device}のコマンドを実行します.

\begin{lstlisting}[language=sh, label=code:check_driver_device, caption=Cheking Motor Driver Device]
ls /dev/ttyACM*
\end{lstlisting}

他にUSBデバイスを接続していない場合は,TF-2MD-R6は\textsf{/dev/ttyACM0}として認識されるはずです.
もしデバイスファイルが1つも表示されなかった場合は,モータードライバに電源が供給されているか,及びPCに正しく接続されているかを確認してください.
また,もしファイルの番号が\textsf{/dev/ttyACM0}ではなく\textsf{/dev/ttyACM1}や\textsf{/dev/ttyACM2}等だった場合は,それに合わせて\textsf{adamr2\_control.launch}のパラメータを変更する必要があります.

\subsection{launchファイルの実行}

デバイスを正しく接続できたら,あとは\textsf{adamr2\_control.launch}を実行すればモータードライバが起動し,移動可能になります.
コード\ref{code:launch_controller}を実行すればコントローラが起動します.

\begin{lstlisting}[language=sh, label=code:launch_controller, caption=launch \textsf{adamr2\_control.launch}]
roslaunch adamr2_control adamr2_control.launch
\end{lstlisting}

コントローラが正しく起動すれば,あとは\textsf{/adamr2/diff\_drive\_controller/cmd\_vel}トピックに速度指令を送ることでロボットが動きます.
手動操縦を行う場合は,\ref{sec:f310_f710_gamepad}で作成した\textsf{joy.launch}を実行すればOKです.

\begin{lstlisting}[language=sh, label=code:launch_joy, caption=launch \textsf{joy.launch}]
roslaunch adamr2_bringup joy.launch
\end{lstlisting}

\end{document}