\documentclass[{../../master}]{subfiles}
\graphicspath{{../../}}  % 個別コンパイル時の画像パスを解決する

\begin{document}

\section{\textsf{ypspur\_launcher.sh}の作成}

\textsf{ypspur\_launcher.sh}は,ROSのプログラムとモータードライバとの通信に必要なYP-Spurのソフトウェア\textsf{ypspur-coordinator}を起動するためのシェルスクリプトです.
\textsf{adamr2\_control}ディレクトリ以下に\textsf{scripts/}ディレクトリを作成し,\textsf{ypspur\_launcher.sh}という名前のファイルを作成します.
そして,コード\ref{code:ypspur_launcher_sh}のように記述します.

\begin{lstlisting}[language=sh, label=code:ypspur_launcher_sh, caption=\textsf{ypspur\_launcher.sh}]
#!/bin/bash

# Execute yp-spur in background
# Argument 1 is the path of parameter file
# Argument 2 is the device file of the motor driver
ypspur-coordinator -p $1 -d $2

echo "Robot parameters [$1] loaded."
\end{lstlisting}

中身は単純で,コマンド引数としてYP-Spurのパラメータファイルとモータードライバのデバイスファイルを受け取り,\textsf{ypspur-coordinator}を実行するだけです.
コメント文に書かれている通り,\textsf{\$1}はパラメータファイル,\textsf{\$2}はデバイスファイルを表します.
このシェルスクリプトは\textsf{adamr2\_control.launch}ファイルから呼び出されます.

\end{document}