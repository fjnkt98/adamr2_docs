\documentclass[{../../master}]{subfiles}
\graphicspath{{../../}}  % 個別コンパイル時の画像パスを解決する

\begin{document}

\subsection{YP-Spurのためのパラメータファイル}

YP-Spurに与えるパラメータファイルも\textsf{adamr2\_control}パッケージに配置します.
\textsf{config/}ディレクトリ内に\textsf{adamr2.param}という名前のファイルを作成し,コード\ref{code:ypspur_param}を記述します.

\begin{lstlisting}[label=code:ypspur_param, caption=\textsf{adamr2.param}]
VERSION 4
COUNT_REV 400
VOLT 24
CYCLE 0.001
GEAR 75
MOTOR_R 0.75
MOTOR_PHASE 3
TORQUE_FINENESS 0.000001
RADIUS[0] 0.07455
RADIUS[1] -0.07455
TREAD 0.383
CONTROL_CYCLE 0.015
TORQUE_MAX 1.0
TORQUE_LIMIT 1.0
MAX_VEL 0.9
MAX_W 3.14
MAX_ACC_V 1.5
MAX_ACC_W 6.28
MAX_CENTRI_ACC 2.45
L_C1 0.01
L_K1 800
L_K2 300
L_K3 200
L_DIST 0.6
INTEGRAL_MAX 0.05
MOTOR_VC 630.0
MOTOR_TC 0.01515
MOTOR_M_INERTIA 0
TIRE_M_INERTIA 0.02
MASS 10
MOMENT_INERTIA 0.1
GAIN_KP 120
GAIN_KI 300
TORQUE_VISCOS 0.00001
TORQUE_NEWTON 0.00200
\end{lstlisting}

このパラメータは移動ロボットプラットフォーム「i-cart mini」\footnote{\url{http://t-frog.com/products/icart_mini/}}のものを参考にして作成しました.
各種パラメータの詳細については,YP-Spurのホームページ(\url{https://www.roboken.iit.tsukuba.ac.jp/platform/wiki/yp-spur/parameter-file})を参考にしてください.

\end{document}