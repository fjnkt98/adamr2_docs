\documentclass[{../../master}]{subfiles}
\graphicspath{{../../}}  % 個別コンパイル時の画像パスを解決する

\begin{document}

\section{オドメトリ調整}

\ref{sec:config_for_diff_drive_controller}にて,車輪の半径やトレッド等,\textsf{diff\_drive\_controller}に与えるパラメータを記述しました.
パラメータの値は設計値をそのまま設定しましたが,実際にロボットを組み立ててみると設計値からはズレが生じます.
それによってオドメトリに誤差が混入してしまう問題があります.

この節では\textsf{diff\_drive\_controller}のパラメータをチューニングすることによるオドメトリ調整の手順を説明します.
あくまで調整方法の例の1つであり,本節で説明する例が絶対的に正しいことを保証するものではありません.

\subsection{差動二輪ロボットの運動学}

ここで差動二輪ロボットの運動学について振り返っておきます.
図\ref{fig:coordinate_system_of_ddmr}に移動ロボットの座標系を示します.

グローバル座標系として$\Sigma_{G}$を,ローカル座標系として$\Sigma_{M}$を定義します.
車輪半径を$R$,ホイールトレッドを$2L$とし,左右の車輪の回転速度をそれぞれ$\dot{\phi_{r}}$,$\dot{\phi_{l}}$[\SI{}{rad/s}]とします.
また,グローバル座標系におけるロボットの位置を$\vec{x} = (x, y, \theta)^T$とします.

ロボットの直進速度を$v$,旋回速度を$\omega$とすると,ローカル座標系におけるロボットの移動速度ベクトルは${}^{M}\dot{\vec{x}} = (v, 0, \omega)^T$,
グローバル座標系におけるロボットの移動速度ベクトルは${}^{G}\dot{\vec{x}} = (v\cos{\theta}, v\sin{\theta}, \omega)^T$となります.

以上より,差動二輪ロボットの運動学は式\ref{eq:ddmr_kinematics}のようになります.

\begin{equation}
  \begin{cases}
    v = R \frac{\dot{\phi_{r}} + \dot{\phi_{l}}}{2} \\
    \omega = R \frac{\dot{\phi_{r}} - \dot{\phi_{l}}}{2L}
  \end{cases}
  \label{eq:ddmr_kinematics}
\end{equation}

\noindent
式\ref{eq:ddmr_kinematics}の行列表示は式\ref{eq:ddmr_kinematics_in_matrix}のようになります.

\begin{equation}
  \begin{pmatrix}
    v \\
    \omega
  \end{pmatrix}
  =
  \begin{pmatrix}
    \frac{R}{2} & \frac{R}{2} \\
    \frac{R}{2L} & -\frac{R}{2L}
  \end{pmatrix}
  \begin{pmatrix}
    \dot{\phi_{r}} \\
    \dot{\phi_{l}}
  \end{pmatrix}
  \label{eq:ddmr_kinematics_in_matrix}
\end{equation}

ホイールオドメトリの式は,式\ref{eq:ddmr_kinematics}の式から求めたロボットの速度を積分したものになります.

\begin{equation}
  \begin{cases}
    \theta (t) = \int_0^t \omega(\tau)d\tau + \theta (t_0) \\
    x(t) = \int_0^t v(\tau)\cos{\theta (\tau)}d\tau + x(t_0) \\
    y(t) = \int_0^t v(\tau)\sin{\theta (\tau)}d\tau + y(t_0)
  \end{cases}
  \label{eq:odometry_equation}
\end{equation}

\noindent
式\ref{eq:odometry_equation}は連続時間での方程式です.
サンプリング時間$\Delta{t}$で式\ref{eq:odometry_equation}を離散化すると,

\begin{equation}
  \begin{cases}
    \theta (t) = \sum_{i=1}^t \omega_{i}\Delta{t} + \theta_{0} \\
    x(t) = \sum_{i=1}^t v_{i}\cos{\theta_{i}} + x_{0} \\
    y(t) = \sum_{i=1}^t v_{i}\sin{\theta_{i}} + y_{0}
  \end{cases}
\end{equation}

\noindent
となります.ただし,$v_{i}$,$\omega_{i}$は時刻$i-1$から$i$までの速度の値です.

\begin{figure}[ht]
  \centering
  \begin{tikzpicture}
    % Origin
    \draw (-5, -5) node[below left]{O};
    % X Axis(global)
    \draw[->, >=stealth, semithick] (-5, -5) -- (5, -5) node[right]{$X_{G}$}; % X軸
    % Y Axis(global)
    \draw[->, >=stealth, semithick] (-5, -5) -- (-5, 5) node[above]{$Y_{G}$}; % Y軸
  
    % Body
    \draw[thick, rotate around={30:(0, 0)}]
      (-0.75, 1) -- (0.75, 1) -- (1.2, 0)
        -- (0.75, -1) -- (-0.75, -1) -- (-0.75, 1);
    % X Axis(local)
    \draw[->, >=stealth, semithick, rotate around={30:(0, 0)}]
      (0, 0) -- (3, 0) node[right]{$X_{M}$};
    % Y Axis(local)
    \draw[->, >=stealth, semithick, rotate around={30:(0, 0)}]
      (0, 0) -- (0, 3) node[left]{$Y_{M}$};
    % Left Wheel
    \draw[thick, rotate around={30:(0, 0)}]
      (-0.6, 1.1) -- (0.6, 1.1) -- (0.6, 1.5) -- (-0.6, 1.5) -- (-0.6, 1.1);
    % Right Wheel
    \draw[thick, rotate around={30:(0, 0)}]
      (-0.6, -1.1) -- (0.6, -1.1) -- (0.6, -1.5) -- (-0.6, -1.5) -- (-0.6, -1.1);
    % Left Wheel Leader Line
    \draw[thick, rotate around={30:(0, 0)}]
      (-0.6, 1.3) -- (-1.6, 1.3);
    % Right Wheel Leader Line
    \draw[thick, rotate around={30:(0, 0)}]
      (-0.6, -1.3) -- (-1.6, -1.3);
    % Tread Dimension Line
    \draw[thick, <->, rotate around={30:(0, 0)}]
      (-1.2, -1.3) -- (-1.2, 1.3);
    \draw[rotate around={30:(0, 0)}] (-1.2, 0.2) node[below left]{$2l$};

    % Wheel Radius Leader Line
    \draw[thick, rotate around={30:(0, 0)}]
      (0, -1.5) -- (0, -2.0);
    \draw[thick, rotate around={30:(0, 0)}]
      (0.6, -1.5) -- (0.6, -2.0);
    \draw[thick, <->,  rotate around={30:(0, 0)}]
      (0, -1.8) -- (0.6, -1.8);
    \draw[rotate around={30:(0, 0)}] (0.15, -1.8) node[below right]{$R$};

    % Center of Robot
    \draw[thick, dashed] (0, 0) -- (0, -5) node[below]{$x$};
    \draw[thick, dashed] (0, 0) -- (-5, 0) node[left]{$y$};
    \draw[thick, dashed] (0, 0) -- (3, 0);
    \draw[thick, ->] (2, 0) arc [start angle=0, delta angle=30, radius=2];
    \draw (2, 0.6) node[right]{$\theta$};
  \end{tikzpicture}
  \caption{Coordinate System of Differential Drive Mobile Robot}
  \label{fig:coordinate_system_of_ddmr}
\end{figure}

\subsection{調整対象のパラメータ}

\subsection{パラメータ調整の手順}

\end{document}