\documentclass[{../../master}]{subfiles}
\graphicspath{{../../}}  % 個別コンパイル時の画像パスを解決する

\begin{document}

\section{SLAM実験の手順}

ADAMR2を使って\textsf{slam\_gmapping}を用いたSLAMによる地図作成を行うことができます.
SLAMで地図作成実験を行うときは,以下の2つのシチュエーションが考えられます.

\begin{itemize}
  \item オンラインSLAM:ロボットを走行させながらリアルタイムに地図作成を行う
  \item オフラインSLAM:ロボットの走行データを保存して,それを再生しながら地図作成を行う
\end{itemize}

本節では2つのシチュエーションに分けて,SLAMの実行手順を説明します.

\subsection{オンラインSLAM}

SLAMを行うには,ロボットのオドメトリ情報とレーザースキャン情報が必要になります.
オドメトリ情報はモータードライバが提供してくれるので,あとはレーザースキャン情報を追加すれば良いということになります.

まずは\ref{sec:teleoperation}で説明したように,ロボットを走行可能な状態にします.
そして,RPLiDARをPCに接続し,新しいターミナルを開いてコード\ref{code:launch_rplidar_for_slam}を実行します.

\begin{lstlisting}[language=sh, label=code:launch_rplidar_for_slam, caption=Launch \textsf{rplidar.launch}]
roslaunch adamr2_bringup rplidar.launch
\end{lstlisting}

RPLiDARが起動して,レーザースキャンのデータを取得できるようになれば準備完了です.
あとは\textsf{slam\_gmapping}を起動するlaunchファイルを実行します.
新しいターミナルを開き,コード\ref{code:launch_gmapping_for_slam}を実行します.

\begin{lstlisting}[language=sh, label=code:launch_gmapping_for_slam, caption= Launch \textsf{gmapping.launch}]
roslaunch adamr2_slam gmapping.launch situation:=real
\end{lstlisting}

注意点として,実験時のシチュエーションを引数に指定しなければなりません.
これは,GazeboシミュレーションでのSLAM実験と,実機による実験とで\textsf{slam\_gmapping}のパラメータを切り替えるための設定です.
\textsf{situation}引数を設定しないと,\textsf{gmapping.launch}を実行することはできません.

\textsf{gmapping.launch}を実行すると,\textsf{slam\_gmapping}ノードと\textsf{rviz}が起動します.
\textsf{rviz}の画面を確認して,トピック等に抜けがないかをチェックしてください.
特にRPLiDARは電流不足によるデータの配信ミスが多いです.

トピックに問題が無ければ,ゲームパッドによってロボットを操縦します.
ロボットの移動に従って,地図がどんどん作成されていくのが確認できます

地図の作成が終わったら,作成した地図を保存しておきます.
地図の保存は\textsf{map\_server}パッケージの\textsf{map\_saver}ノードによって行えます.
\footnote{\textsf{map\_server}パッケージには\textsf{map\_server}ノードと\textsf{map\_saver}ノードが存在します.名前が似ていて非常に紛らわしいので,間違えないように注意してください.}

地図を保存したいディレクトリに移動し,コード\ref{code:save_map}を実行します.

\begin{lstlisting}[language=sh, label=code:save_map, caption=Save Map]
rosrun map_server map_saver
\end{lstlisting}

コード\ref{code:save_map}を実行すると,カレントディレクトリに\textsf{map.pgm}と\textsf{map.yaml}の2つのファイルが生成されます.

地図の保存が完了したら,ロボットを終了させます.

\subsection{オフラインSLAM}



\end{document}