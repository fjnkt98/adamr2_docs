\documentclass[{../../master}]{subfiles}
\graphicspath{{../../}}  % 個別コンパイル時の画像パスを解決する

\begin{document}

\section{SLAM実験の手順}

ADAMR2を使って\textsf{slam\_gmapping}を用いたSLAMによる地図作成を行うことができます.
SLAMで地図作成実験を行うときは,以下の2つのシチュエーションが考えられます.

\begin{itemize}
  \item オンラインSLAM:ロボットを走行させながらリアルタイムに地図作成を行う
  \item オフラインSLAM:ロボットの走行データを保存して,それを再生しながら地図作成を行う
\end{itemize}

本節では2つのシチュエーションに分けて,SLAMの実行手順を説明します.

\subsection{オンラインSLAM}

SLAMを行うには,ロボットのオドメトリ情報とレーザースキャン情報が必要になります.
オドメトリ情報はモータードライバが提供してくれるので,あとはレーザースキャン情報を追加すれば良いということになります.

まずは\ref{sec:teleoperation}で説明したように,ロボットを走行可能な状態にします.
そして,RPLiDARをPCに接続し,新しいターミナルを開いてコード\ref{code:launch_rplidar_for_slam}を実行します.

\begin{lstlisting}[language=sh, label=code:launch_rplidar_for_slam, caption=Launch \textsf{rplidar.launch}]
roslaunch adamr2_bringup rplidar.launch
\end{lstlisting}

RPLiDARが起動して,レーザースキャンのデータを取得できるようになれば準備完了です.
あとは\textsf{slam\_gmapping}を起動するlaunchファイルを実行します.
新しいターミナルを開き,コード\ref{code:launch_gmapping_for_slam}を実行します.

\begin{lstlisting}[language=sh, label=code:launch_gmapping_for_slam, caption= Launch \textsf{gmapping.launch}]
roslaunch adamr2_slam gmapping.launch situation:=real
\end{lstlisting}

注意点として,実験時のシチュエーションを引数に指定しなければなりません.
これは,GazeboシミュレーションでのSLAM実験と,実機による実験とで\textsf{slam\_gmapping}のパラメータを切り替えるための設定です.
\textsf{situation}引数を設定しないと,\textsf{gmapping.launch}を実行することはできません.

\textsf{gmapping.launch}を実行すると,\textsf{slam\_gmapping}ノードと\textsf{rviz}が起動します.
\textsf{rviz}の画面を確認して,トピック等に抜けがないかをチェックしてください.
特にRPLiDARは電流不足によるデータの配信ミスが多いです.

トピックに問題が無ければ,ゲームパッドによってロボットを操縦します.
ロボットの移動に従って,地図がどんどん作成されていくのが確認できます

地図の作成が終わったら,作成した地図を保存しておきます.
地図の保存は\textsf{map\_server}パッケージの\textsf{map\_saver}ノードによって行えます.
\footnote{\textsf{map\_server}パッケージには\textsf{map\_server}ノードと\textsf{map\_saver}ノードが存在します.名前が似ていて非常に紛らわしいので,間違えないように注意してください.}

地図を保存したいディレクトリに移動し,コード\ref{code:save_map}を実行します.

\begin{lstlisting}[language=sh, label=code:save_map, caption=Save Map]
rosrun map_server map_saver
\end{lstlisting}

コード\ref{code:save_map}を実行すると,カレントディレクトリに\textsf{map.pgm}と\textsf{map.yaml}の2つのファイルが生成されます.

地図の保存が完了したら,ロボットを終了させます.

\subsection{オフラインSLAM}

オフラインでSLAMを実行する場合は,まずロボットの走行データをBagファイルに保存する必要があります.
前小節と同様にロボットとLiDARを起動し,走行データを集めます.

\subsubsection{ヘルパースクリプトを使用した走行データの記録}

走行データの記録は\textsf{rosbag}コマンドを使用するのですが,ADAMR2ではSLAM用のデータを一括で記録するためのヘルパースクリプトを作ったので,それを使います.
ヘルパースクリプトは\textsf{adamr2\_ros1\_packages/utility/bag\_tools}ディレクトリに置かれています.
SLAM用のデータを集めるスクリプトは\textsf{slam\_real.sh}です.
\textsf{slam\_real.sh}の中身はコード\ref{code:slam_real_sh}のようになっています.

\begin{lstlisting}[language=sh, label=code:slam_real_sh, caption=\textsf{slam\_real.sh}]
#!/bin/bash
set -e

TODAY=$(date "+%Y%m%d")

TARGET_DIR="/home/${USER}/adamr2_experiments/real/slam/${TODAY}"

if [ ! -e $TARGET_DIR ]; then
  mkdir -p $TARGET_DIR
  echo "-- Target directory was successfully generated. --"
fi

cd $TARGET_DIR

DIR_NUM=$(ls -l | grep ^d | wc -l)
TITLE_NUM=$((++DIR_NUM))

mkdir exp$((TITLE_NUM))
cd exp$((TITLE_NUM))

mkdir bag && cd bag

rosbag record /tf \
              /tf_static \
              /adamr2/camera/depth/color/points \
              /adamr2/camera/color/image_raw \
              /adamr2/scan \
              /adamr2/diff_drive_controller/odom
\end{lstlisting}

\textsf{slam\_real.sh}はホームディレクトリに\textsf{adamr2\_experiments}というディレクトリを作り,実行時の日付を名前としたディレクトリを作ります.
また,\textsf{slam\_real.sh}が実行される度にディレクトリを作るようになっているので,ディレクトリ構造がすっきりするようになっています.

\textsf{slam\_real.sh}を実行するには,スクリプトが置かれているディレクトリに移動し,コード\ref{code:run_slam_real_sh}を実行します.

\begin{lstlisting}[language=sh, label=code:run_slam_real_sh, caption=Run \textsf{slam\_real.sh}]
bash ./slam_real.sh
\end{lstlisting}

\textsf{slam\_real.sh}を実行すると\textsf{rosbag}によるデータの記録が始まります.
そのままロボットを走行させ,走行データを収集します.

記録をストップするには,\textsf{slam\_real.sh}を実行したターミナルで\textsf{Ctrl+C}を押せば終了します.

\subsubsection{bagファイルによるSLAM}

bagファイルが記録できたら,それを使ってSLAMを行います.
オフラインSLAMではロボットを使う必要は無いため,ここからの作業ではロボットの電源を切っておいて構いません.

まず新しいターミナルを開き,ROS Masterを起動します.

\begin{lstlisting}[language=sh, caption=Execute \textsf{roscore}]
roscore
\end{lstlisting}

更に新しいターミナルを開き,\textsf{use\_sim\_time}パラメータをtrueに設定します.
これは,\textsf{rosbag}のファイルを再生して処理を行う際に必要となるオプションです.

\begin{lstlisting}[language=sh, caption=Set \textsf{use\_sim\_time} to true]
rosparam use_sim_time true
\end{lstlisting}

また,ロボット内部のTF情報をROS側が見れるようにするため,\textsf{robot\_description}パラメータにロボットのURDFを登録する必要があります.
\textsf{adamr2\_description}パッケージにそれ用のlaunchファイルを準備しているので,それを実行します.

\begin{lstlisting}[language=sh, caption=Launch \textsf{adamr2\_description.launch}]
roslaunch adamr2_description adamr2_description.launch
\end{lstlisting}

\noindent
何故URDFの登録が必要かというと,ロボットのベースリンクからLiDARリンクまでの座標変換が必要になるのと,ロボットのモデルを\textsf{rviz}で表示するためです.

更にもう1つのターミナルを開き,\textsf{slam\_gmapping}を実行します.
先ほど作成したlaunchファイルを実行します.
\textsf{slam\_gmapping}はデータが送られない限り処理を行わないので,Bagファイルを再生する前に起動しておきます.
\textsf{situation}引数の設定を忘れないようにしましょう.

\begin{lstlisting}[language=sh, caption=Run \textsf{gmapping.launch}]
roslaunch adamr2_slam gmapping.launch situation:=real
\end{lstlisting}

最後にBagファイルを再生します.
ターミナルを新しく開き,Bagファイルがあるディレクトリに移動し,コード\ref{code:replay_bag_file_for_slam}を実行します.

\begin{lstlisting}[language=sh, label=code:replay_bag_file_for_slam, caption=Replay \textsf{slam.bag}]
rosbag play --clock XXXXXXXXXXXXX.bag
\end{lstlisting}

Bagファイル名は適切な名前を入力してください.
\textsf{Tab}キーを押せば保管されるはずなので,入力は簡単です.

Bagファイルの再生に伴い,SLAMが実行され,地図が生成されます.
Bagファイルの再生が終了したら地図を保存しましょう.
地図の保存場所は,Bagファイルがあるディレクトリと紐付けておくのが良いでしょう.

\end{document}