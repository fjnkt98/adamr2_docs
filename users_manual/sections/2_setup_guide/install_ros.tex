\documentclass[{../../master}]{subfiles}
\graphicspath{{../..}}  % 個別コンパイル時の画像パスを解決する

\begin{document}

\section{ROS Melodicのインストール}

ROS Melodic Moreniaをインストールします.
インストール手順はROS Wiki\footnote{\url{http://wiki.ros.org/melodic/Installation/Ubuntu}}に書かれている内容と同じですが,ビルドシステムに\textsf{catkin}ビルドツールを使用する点が異なります.

\subsection{ROS本体のインストール}

\textsf{Ctrl + Alt + T}を押してターミナルを開き,以下のコマンドを順番に実行します.

\noindent
【注意】PDFのコードブロックはコピーペーストした際に文章が崩れる可能性があります.コマンドのコピペはROS Wikiから行うことをお勧めします.

\begin{lstlisting}[language=sh, caption=リポジトリの鍵の取得]
sudo sh -c 'echo "deb http://packages.ros.org/ros/ubuntu $(lsb_release -sc) main" \
  > /etc/apt/sources.list.d/ros-latest.list'
\end{lstlisting}

\begin{lstlisting}[language=sh, caption=鍵の登録]
sudo apt-key adv --keyserver 'hkp://keyserver.ubuntu.com:80' --recv-key \
  C1CF6E31E6BADE8868B172B4F42ED6FBAB17C654
\end{lstlisting}

\begin{lstlisting}[language=sh, caption=パッケージリストのアップデート]
sudo apt update
\end{lstlisting}

ここではデスクトップ版のROSをインストールします.

\begin{lstlisting}[language=sh, caption=ROS本体のインストール]
sudo apt install ros-melodic-desktop-full
\end{lstlisting}

\begin{lstlisting}[language=sh, caption=ROSのセットアップスクリプトの読み込みを自動化する]
echo "source /opt/ros/melodic/setup.bash" >> ~/.bashrc
source ~/.bashrc
\end{lstlisting}

\begin{lstlisting}[language=sh, caption=ユーティリティツールのインストール]
sudo apt install \
  python-rosdep \
  python-rosinstall \
  python-rosinstall-generator \
  python-wstool \
  build-essential
\end{lstlisting}

\begin{lstlisting}[language=sh, caption=\textsf{rosdep}のインストール]
sudo rosdep init
rosdep update
\end{lstlisting}

\subsection{\textsf{catkin}ビルドツールのインストール}

ROS1のモダンなビルドツールとして,Catkin Command Line Tools\footnote{\url{https://catkin-tools.readthedocs.io/en/latest/}}が開発されています.

ROS WikiやROSの教本などでは\textsf{catkin\_make}コマンドを使ってワークスペースの設定やパッケージのビルドを行っていますが,\textsf{catkin}ビルドツールの方が使い勝手が良いです.
これからROS1を始める際は\textsf{catkin\_make}コマンドでなくこちらを使うことをおすすめします.

Catkin Command Line ToolsはデフォルトでROSに含まれていないので,追加でインストールする必要があります.

\begin{lstlisting}[language=sh, caption=\textsf{catkin}ビルドツールのインストール]
sudo apt install python-catkin-tools
\end{lstlisting}

\noindent
【注意】\textsf{catkin}コマンドと\textsf{catkin\_make}コマンドの間には互換性がありません.
そのため,\textsf{catkin\_make}コマンドで初期化したワークスペースを\textsf{catkin}コマンドでビルドすることはできません.
もし既に\textsf{catkin\_make}コマンドでワークスペースを初期化している場合には,\textsf{catkin\_ws}ディレクトリにある\textsf{src/}以外のディレクトリを削除してから\textsf{catkin}コマンドでビルドし直してください.

\subsection{ワークスペースの初期化}

\textsf{catkin}コマンドを使って,ROSのワークスペースを初期化します.

\begin{lstlisting}[language=sh, caption=ワークスペースの初期化]
mkdir -p ~/catkin_ws/src
cd ~/catkin_ws
catkin init
\end{lstlisting}

これでROSのインストール作業は終了です.
次に,ADAMR2用のROSパッケージをインストールする作業を行います.

\end{document}