\documentclass[{../../master}]{subfiles}
\graphicspath{{../../}}  % 個別コンパイル時の画像パスを解決する

\begin{document}
  \section{電装の繋ぎ方}

  鉛蓄電池,スイッチ,モータードライバ等のロボットの電装を接続します.

  \section{部品一覧}
  
  電装の配線には以下の部品を使用します.

  \subsection{電装部品一覧}

  \begin{table}[ht]
    \begin{center}
      \begin{tabular}{|l|l|c|}
        \hline
          Name & Model Number & Quantity \\ \hline
          鉛蓄電池 12V 20Ah & WP20-12IE & 2 \\ \hline
          端子台 & ATK-30-6P & 1  \\ \hline
          メイン電源用スイッチ & 273-025 & 1  \\ \hline
          DCコンバータ用スイッチ & RB625011FF-10P & 1  \\ \hline
          DCコンバータ 12V-19V & a16052500ux0037 & 1  \\ \hline
          DCプラグ & 761KS15 & 1  \\ \hline
          T-Frog モータードライバ & TF-2MD3-R6 & 1  \\ \hline
          T-Frog 電源基板 & TF-PW36-5/12M & 1  \\ \hline
        \end{tabular}
    \end{center}
    \caption{List of Electrical Components}
    \label{tab:list_of_electrical_components}
  \end{table}

  \subsection{配線一覧}

  \begin{table}[ht]
    \begin{center}
      \begin{tabular}{|l|l|}
        \hline
        Name & Model Number \\ \hline
        ニチフ 銅線用絶縁被覆付圧着端子丸型 & TMEV1.25-3 \\ \hline
        差込み型接続端子 187シリーズ(バリュー品) メス(嵌合部絶縁型) & MTR-480809-FA \\ \hline
        絶縁付圧着端子 Y型 & F1.25-5 \\ \hline
        通信機器用ビニル電線 KVシリーズ & KV 1.25SQ アカ-200 \\ \hline
        通信機器用ビニル電線 KVシリーズ & KV 1.25SQ クロ-200 \\ \hline
      \end{tabular}
    \end{center}
    \caption{List of Wiring Components}
    \label{tab:list_of_wiring_components}
  \end{table}

  \section{接続図}

  電装部品の接続図を図\ref{fig:electrical_connection_diagram}に示します.

  \begin{figure}[ht]
    \centering
    \includegraphics[width=100truemm, clip]{images/electrical_connections.png}
    \caption{Electrical Connection Diagram}
    \label{fig:electrical_connection_diagram}
  \end{figure}

  \section{端子の接続方法の詳細}

  \subsection{鉛蓄電池}
  鉛蓄電池WP20-12IEはM5ねじの接続端子を持っているので,Y型圧着端子をねじ止めすることで接続することができます.  
  ねじの締結には8mm幅のレンチを使用できます.感電やショートに十分注意して作業してください.

  \begin{figure}[ht]
    \centering
    \includegraphics[width=65truemm, clip]{images/pb_battery.jpg}
    \caption{Detail of How to Connect Pb-Battery}
    \label{fig:pb_battery}
  \end{figure}

  \subsection{端子台}
  端子台ATK-30-6PはM5ねじの接続端子を持っているので,Y型圧着端子をねじ止めすることで接続できます.

  1つの端子に2つの圧着端子を接続することもできるので,圧着端子を両端につけたケーブルを使って2つの端子列をショートさせることができます.
  上の接続例でも,GNDや12Vの箇所でこのような接続方法をとっています.

  \begin{figure}[ht]
    \centering
    \includegraphics[width=65truemm, clip]{images/terminal.jpg}
    \caption{Detail of How to Connect Screw Terminal Block}
    \label{fig:terminal}
  \end{figure}

  \subsection{スイッチ}
  ADAMR2で使用しているスイッチは,いずれも187シリーズの差込型接続端子(ファストン端子)を持っています.
  接続には187シリーズの差込型端子(メス)を付けたケーブルを使用する必要があります.

  差込型接続端子ははんだ付け無しで何度も取り外しができますが,付け直す度に端子が摩擦で削れて接続が緩くなる可能性があります.
  取り外しはできるだけ最小限に抑えるよう注意してください.

  \begin{figure}[ht]
    \centering
    \includegraphics[width=65truemm, clip]{images/switch.jpg}
    \caption{Detail of How to Connect Switch}
    \label{fig:switch}
  \end{figure}

  \subsection{T-Frog電源基板}
  T-Frog電源基板は,M3ねじ止めターミナルを持っています.
  接続には内径φ3.2の圧着端子を使用します.ADAMR2では丸形圧着端子を使用しています.

  \begin{figure}[ht]
    \centering
    \includegraphics[width=65truemm, clip]{images/t-frog.jpg}
    \caption{Detail of How to Connect T-Frog Power Supply Board}
    \label{fig:t-frog}
  \end{figure}
\end{document}